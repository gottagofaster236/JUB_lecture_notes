\pagebreak
\begin{definition}[Algebra]
    Let $X$ be a non-empty set. $\Omega \subset 2^X$ is an algebra,
    if:
    \begin{enumerate}
        \item {
            $X \in \Omega$;
        }
        \item {
            $\Omega$ is closed under the formation of complements in $X$
             and \textit{finite} unions.
        }
    \end{enumerate}
    \begin{remark}
        It follows that $\Omega$ is also closed under intersections:
        \[
            \bigl(X_1^C \cup \dots \cup X_n^C\bigr)^C = X_1 \cap \dots \cap X_n
        \]
    \end{remark}
\end{definition}

\begin{definition}[$\sigma$-algebra]
    Let $X$ be a non-empty set. $\Omega \subset 2^X$ is a $\sigma$-algebra,
    if:
    \begin{enumerate}
        \item {
            $X \in \Omega$;
        }
        \item {
            $\Omega$ is closed under the formation of complements in $X$
             and \textit{countable} unions.
        }
    \end{enumerate}
\end{definition}
\begin{remark}
    Every $\sigma$-algebra is an algebra, but not vice versa.
\end{remark}

\begin{corollary}
    The collection $\mathcal{M}$ of all measurable subsets
     of $\mathbb{R}$ is an algebra.
\end{corollary}
\begin{proof}
    For the proof, we'll need to show that:
    \begin{enumerate}
        \item {
            $\mathbb{R}$ is measurable.
            \[
                m^*(A) = m^*(A \cap \mathbb{R}) + m^*(A \cap \mathbb{R}^C) = 
                m^*(A) + m^*(\emptyset)
            \]
        }
        \item {
            It is closed under complements. It follows from the symmetry of the
            \hyperref[def:measurable]{definition
            of a measurable set}.
        }
        \item {
            It is closed under unions.  We \hyperref[the:measurableUnion]{have already proved} this one.
        }
    \end{enumerate}
\end{proof}

\begin{proposition}
    \label{prop:disjointIntersectionSum}
    $\{E_k\}_1^n$ --- disjoint measurable sets. Then for every set $A$
    \[
        m^*\Bigl(A \cap \Bigl[\bigcup_1^n E_k\Bigr]\Bigr) = \sum_1^n m^*(A \cap E_k)
    \]
    In particular, for $A = \mathbb{R}$ we have
    \[
        m^*\Bigl(\bigcup_1^n E_k\Bigr) = \sum_1^n m^*(E_k)
    \]
\end{proposition}
\begin{proof}
    Induction on $n$.

    Base $n=1$ is obvious.

    Step $n - 1 \to n$. Take $\hat{A} := A \cap \bigl[ \bigcup_1^n E_k \bigr]$.
    Then \[\hat{A} \cap E_n = A \cap E_n \]
    We also have \[\hat{A} \cap E_n^C = A \cap \Bigl[ \bigcup_1^{n - 1} E_k \Bigr]\]
    That is true, as intersecting 
    with $E_n^C$ is equivalent to subtracting $E_n$ from $\hat{A}$, and since
    $\{E_k\}$ are disjoint, no other parts of $\hat{A}$ except $E_n$ will be removed. Then:
    \begin{align*}
        &m^*(\hat{A}) \overset{E_n \text{ is measurable}}{=\joinrel=} m^*(\hat{A} \cap E_n) + m^*(\hat{A} \cap E_n^C)
        =
        \\&
        = m^*(A \cap E_n) + m^*\Bigl(A \cap \Bigl[\bigcup_{1}^{n-1} E_k\Bigr]\Bigr)
        \overset{\text{induction}}{=} m^*(A \cap E_n) + \sum_{1}^{n-1} m^*(A \cap E_k)
    \end{align*}
\end{proof}

\begin{proposition}
    \label{prop:unionOfCountable}
    The union of a countable collection of measurable sets is the union
    of a countable collection of \textit{disjoint} measurable sets.
\end{proposition}
\begin{proof}
    If $A = \cup_1^\infty A_k$, define $\hat{A}_1 := A_1$ and
    $\hat{A}_k := A_k \setminus \cup_1^{k-1} A_j$.
    As $\mathcal{M}$ is an algebra, all $\hat{A}_k$ are measurable, and
    $A = \sqcup_1^\infty \hat{A}_k$, which is what we wanted.
\end{proof}

\begin{theorem}
    $\mathcal{M}$ is a $\sigma$-algebra.
\end{theorem}
\begin{proof}
    We need to show that if all $\{E_k\}_1^\infty$ are measurable sets,
    then $E = \cup_1^\infty E_k$ is measurable. 
    By Proposition~\ref{prop:unionOfCountable}, without the loss of generality,
    assume that $E_k$ are all pairwise disjoint.
    Let $F_n := \cup_1^n E_k$, then $F_n \in \mathcal{M}$
    (as a finite union). As $F_n \subset E$, we have $E^C \subset F_n^C$.

    Let $A$ be any set. Then:
    \begin{align*}
        &
        m^*(A) = m^*(A \cap F_n) + m^*(A \cap F_n^C) \ge
        m^*(A \cap F_n) + m^*(A \cap E^C)
        \overset{\text{Proposition~\ref{prop:disjointIntersectionSum}}}{=}
        \\&=
        \sum_1^n m^*(A \cap E_k) + m^*(A \cap E_C)
    \end{align*}
    Now take $n \to \infty$:
    \begin{align*}
        m(A) \ge \sum_1^\infty m^*(A \cap E_k) + m^*(A \cap E_C)
        \overset{\text{\hyperref[the:countableSubadditivity]{countable subadditivity}}}{\ge}
        m^*(A \cap E) + m^*(A \cap E^C)
    \end{align*}
    Now we have the inequality in the difficult direction. The inequality in the other direction
    is obvious (again, from countable subadditivity).
\end{proof}

\begin{proposition}[Countable additivity]
    If $\{E_k\}_1^\infty \subset \mathcal{M}$ --- collection of
    disjoint sets, then  $\cup_1^\infty E_k \in \mathcal{M}$
    and
    \[ m^*\Bigl(\bigcup_1^\infty E_k\Bigr) = \sum_1^\infty m^*(E_k) \] 
\end{proposition}
\begin{proof}
    We know that:
    \begin{enumerate}
        \item {
            \[ m^*\Bigl( \bigcup_1^\infty E_k \Bigr) \le
            \sum_1^\infty m^*(E_k) \text{ (countable \textit{sub}additivity)}\]
        }
        \item {
            \[ m^*\Bigl(\bigcup_1^\infty E_k\Bigr) \ge
            m^*\Bigl( \bigcup_1^n E_k \Bigr) \overset{\text{Proposition~\ref{prop:disjointIntersectionSum}}}{=}
            \sum_1^n m^*(E_k) \]
            Take $n \to \infty$, then
            \[ m^*\Bigl( \bigcup_1^\infty E_k \Bigr) \ge 
            \sum_1^\infty m^*(E_k) \]
            Which is what we wanted.
        }
    \end{enumerate}
\end{proof}

\begin{definition}
    The restriction of $m^*$ on $\mathcal{M}$ is called the
    Lebesgue measure and denoted by $m$.
    \[ m(E) := m^*(E) \quad \forall E \in \mathcal{M} \]
\end{definition}
\begin{definition}
    If $X$ is a non-empty set and $\mathcal{A}$ is a $\sigma$-algebra on $X$,
    then any function $\mu : \mathcal{A} \to [0, +\infty]$ is called the measure
    on $(X, \mathcal{A})$, if:
    \begin{enumerate}
        \item {
            $\mu(\emptyset) = 0$.
        }
        \item {
            $\mu$ is countable additive.
        }
    \end{enumerate}
\end{definition}
