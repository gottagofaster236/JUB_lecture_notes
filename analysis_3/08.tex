
\begin{theorem}[Tietze]
    If $E \in \mathbb{R}$, $E$ is closed, then any continuous
    function on $E$ extends to a continuous function on $\mathbb{R}$.
\end{theorem}
\begin{proof}[Proof idea.]
    The idea is that if $E$ is closed, then its complement, i.e.
    the set where $f$ is undefined, is open.
    As we know, an open set on $\mathbb{R}$ can be represented as a disjoint
    union of a countable number of intervals.
    For every such interval we can simply connect the values of $f$
    on its endpoints with a straight line.

    The technicality that will arise is the case if we have a countable
    number of intervals that accumulate at a point --- that can be left 
    as an exercise.
\end{proof}

\begin{definition}
    Let $(X, \mathcal{A})$, $(Y, \mathcal{D})$ be measurable spaces.
    $f : X \to Y$ is called $(A, \mathcal{D})$-measurable, if
    for every $E \in \mathcal{D}$ we have $f^{-1}(E) \in \mathcal{A}$.
\end{definition}
\begin{proposition}
    Let $f: X \to Y$ be any function. $\mathcal{A}$ is a $\sigma$-algebra on $X$.
    Then the family of sets
    \[ f_*(\mathcal{A}) = \{E \subset Y \mid f^{-1}(E) \in \mathcal{A} \} \]
    is a $\sigma$-algebra on Y.
\end{proposition}
\begin{proof}
    To prove that $f_*(\mathcal{A})$ is a $\sigma$-algebra, we have to prove 3
    properties from the definition of a $\sigma$-algebra.
    \begin{enumerate}
        \item {
            $Y \in f_*(\mathcal{A})$, since $f^{-1}(Y) = X$.
            And $X \in \mathcal{A}$, since $\mathcal{A}$ is a $\sigma$-algebra.
        }
        \item {
            $f_*(\mathcal{A})$ is closed under complements.
            If $E \in f_*(\mathcal{A})$, then
            $f^{-1}(E^C) = \bigl(f^{-1}(E)\bigr)^C$.
            But $f^{-1}(E) \in \mathcal{A}$ by definition of $f_*(\mathcal{A})$.
            Therefore, $\bigl(f^{-1}(E)\bigr)^C \in \mathcal{A}$, since $\mathcal{A}$ is a $\sigma$-algebra.
            Therefore, $f^{-1}(E^C) \in \mathcal{A}$, and thus $E^C \in f_*(\mathcal{A})$
            by definition of $f_*(\mathcal{A})$.
        }
        \item {
            $f_*(\mathcal{A})$ is closed under countable unions.
            If $\{E_n\} \subset f_*(\mathcal{A})$, then 
            \[
                f^{-1}\Bigl(\bigcup_{k=1}^\infty E_k\Bigr) = 
                \bigcup_{k=1}^\infty f^{-1}(E_k) \in \mathcal{A} \implies
                \bigcup_{k=1}^\infty E_k \in f_*(\mathcal{A})
            \]
        }
    \end{enumerate}
\end{proof}
\begin{observation}
    If $f_*(\mathcal{A})$ contains a generating set of $\mathcal{D}$, then
    $f$ is $(\mathcal{A}, \mathcal{D})$-measurable.
\end{observation}
\begin{proof}
    If $f_*(\mathcal{A})$ contains a generating set of $\mathcal{D}$, then, 
    since it is a $\sigma$-algebra, it must contain the whole $\mathcal{D}$. That 
    means that the preimages of all sets from $\mathcal{D}$ lie in 
    $\mathcal{A}$, which is the definition of $(\mathcal{A}, \mathcal{D})$-measurability.
\end{proof}

\begin{corollary}
    \label{cor:measurableSpacesADMeasurable}
    Let $(X, \mathcal{A})$, $(Y, \mathcal{D})$ be measurable spaces.
    $\mathcal{E} \subset \mathcal{D}$ generates $\mathcal{D}$. Then
    $f : X \to Y$ is $(\mathcal{A}, \mathcal{D})$-measurable if and only if
    $\forall E \in \mathcal{E}: f^{-1}(E) \in \mathcal{A}$.
\end{corollary}
\begin{proof}
    This is just a restatement of the observation above.
\end{proof}

\begin{corollary}
    Let $X$, $Y$ be topological spaces. If $f: X \to Y$ is continuous,
    then $f$ is $(\mathcal{B}_X, \mathcal{B}_Y)$-measurable.
\end{corollary}
\begin{proof}
    Take $\mathcal{E}$ to be the set of all open sets in $Y$. 
    $\mathcal{E}$ is a generating set of $\mathcal{B}_Y$. Then, according
    to the Corollary~\ref{cor:measurableSpacesADMeasurable}, we have to prove that the inverse of every set in $\mathcal{E}$
    is in $\mathcal{B}_X$. But $f^{-1}$ of every open set is open by the
    definition of a continuous function, and every open set in $X$ 
    is contained in $\mathcal{B}_X$ (since the Borel $\sigma$-algebra
    is by definition the smallest $\sigma$-algebra containing all open sets).
\end{proof}
\begin{proof}[Proof of Proposition~\ref{pro:everyBorIsMeasurable}]
    Let $X = E$, $Y = \mathbb{R}$ in the previous corollary.
    If $f : E \to \mathbb{R}$ is continuous,
    then $f$ is $(\mathcal{B}_E, \mathcal{B}_\mathbb{R})$-measurable.
    But $\mathcal{B}_E \subset \mathcal{M}$. So, the preimage of any 
    set in $\mathcal{B}_\mathbb{R}$ will be in $\mathcal{M}$.
\end{proof}

\subsection{Measurable functions in $\overline{\mathbb{R}}$}
\begin{definition}
    $\overline{\mathbb{R}} \coloneqq \mathbb{R} \cup \{+\infty, -\infty\}$.
    The topology on $\overline{\mathbb{R}}$ is defined as follow:
    open sets are all open sets in $\mathbb{R}$, all intervals of the form
    $(a, +\infty]$, $[-\infty, a)$, as well as all possible unions (not neccesarily countable)
    of the aforementioned sets.
\end{definition}
\begin{definition}
    We can say that a sequence converges to $+\infty$ or $-\infty$, if for any open set
    containing this infinity there exists such an index of the sequence, 
    starting from which all elements of the sequence are contained in that set.
\end{definition}
\begin{definition}
    Let $E \in \mathbb{R}$ be measurable. Then $f : E \to \overline{\mathbb{R}}$
    is \textit{(Lebesgue) measurable}\\ if it is $(\mathcal{M}, \mathcal{B}_\mathbb{R})$-measurable.
\end{definition}
\begin{example}[1]
    All continuous functions are Lebesgue measurable.
\end{example}
\begin{example}[2]
    The Dirichlet function (1 for $\mathbb{Q}$, 0 for $\mathbb{R} \setminus \mathbb{Q}$)
    is Lebesgue measurable. 
    That's because every preimage for the Dirichlet function
    is either $\mathbb{R}$, $\mathbb{Q}$, or $\mathbb{R} \setminus \mathbb{Q}$, and
    every one of those is measurable.
\end{example}
\begin{corollary}
    Let $E \in \mathcal{M}$, $f : E \to \overline{\mathbb{R}}$ is measurable if and only
    if on the of the following holds:
    \begin{enumerate}
        \item {
            For every open set $O \subset \overline{\mathbb{R}}: f^{-1}(O) \in \mathcal{M}$.
        }
        \item {
            $\forall c \in \overline{\mathbb{R}} \implies
            f^{-1}\bigl((c, +\infty]\bigr) \in \mathcal{M}$.
        }
        \item {
            $\forall c \in \overline{\mathbb{R}} \implies
            f^{-1}\bigl([c, +\infty]\bigr) \in \mathcal{M}$.
        }
        \item {
            $\forall c \in \overline{\mathbb{R}} \implies
            f^{-1}\bigl([-\infty, c)\bigr) \in \mathcal{M}$.
        }
        \item {
            $\forall c \in \overline{\mathbb{R}} \implies
            f^{-1}\bigl([-\infty, c]\bigr) \in \mathcal{M}$.
        }
    \end{enumerate}
\end{corollary}