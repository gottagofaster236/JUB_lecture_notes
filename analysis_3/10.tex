\pagebreak
\begin{corollary}
    For every $p > 0$ and $f$ --- measurable function, 
    $\abs{f}^p$ is measurable.
\end{corollary}
\begin{proof}
    $\abs{x}$ and $x^p$ are continuous as functions from $\mathbb{R}$ to $\mathbb{R}$.
    Therefore, $\abs{f}^p$ is continuous as a composition.
\end{proof}
\begin{remark}
    If $p < 0$, then $x^p$ is not continuous at 0. But we can prove that it is
    as function from $[0, +\infty]$ to $\overline{\mathbb{R}}$.
\end{remark}

\begin{definition}
    $f_1, \dots, f_n : E \to \overline{\mathbb{R}}$. Define
    \[ \max\{f_1, \dots, f_n\}(x) \coloneqq \max \{ f_1(x), \dots, f_n(x) \} \]
\end{definition}
\begin{proposition}
    \label{prop:maxMeasurable}
    If $f_1, \dots, f_n : E \to \overline{\mathbb{R}}$ are measurable,
    then $\max\{f_1, \dots, f_n\}$ is measurable and
    $\min\{f_1, \dots, f_n\}$ is measurable.
\end{proposition}
\begin{proof}
    For all $c \in \mathbb{R}$, 
    \[
        \max\{f_1, \dots, f_n\}^{-1}\bigl((c, +\infty]\bigr) =
        \bigcup_{k=1}^n f_k^{-1}\bigl((c, +\infty]\bigr)
    \]
    which is measurable as a finite union of measurable sets.

    (Meaning of the formula: the maximum is greater than $c$ at a point
    if at least one of the individual functions is greater than $c$ at that point.)
\end{proof}
\begin{remark}
    In principle, $\max$ can be turned to $\sup$ or even $\limsup$.

    For supremum, you would consider a countable union, and for $\limsup$
    you would consider a countable intersection, both of which don't
    change the measurability of sets.
\end{remark}
\begin{definition}
    For a sequence $\{a_k\}$, 
    \[ 
        \limsup{a_n} = \lim_{k \to \infty} \sup\{a_k, a_{k+1}, \dots\}
    \]
\end{definition}

\begin{corollary}
    If $f$ is measurable, then 
    $f^+(x) = \max\{f, 0\}$ (called the \textit{positive part} of $f$) and
    $f^-(x) = \min\{f, 0\}$ (called the \textit{negative part} of $f$)
    are measurable.
\end{corollary}
\begin{proof}
    This is the direct sequence of Proposition~\ref{prop:maxMeasurable}.
\end{proof}

\begin{definition}[Types of convergence]
    $f_n : E \to \overline{\mathbb{R}}$.

    Then the sequence of functions $\{f_n\}$ converges to $f$:
    \begin{enumerate}
        \item {
            \textit{pointwise}, if
             $\lim_{n \to \infty} f_n(x) = f(x),\ \forall x \in E$
        }
        \item {
            \textit{pointwise almost everywhere of $E$} if
            $\lim_{n \to \infty} f_n(x) = f(x)$ for almost every $x \in E$
            (i.e., for every $x$ outside a subset of measure 0).
        }
        \item {
            \textit{uniformly}, if for every $\varepsilon > 0$
            there exists $N \in \mathbb{N}$, such that
            $\abs{f_{n}(x) - f(x)} < \varepsilon$ for any $n > N,\ x \in E$. 
        }
    \end{enumerate}
\end{definition}
\begin{remark}
    Uniform convergence is the strongest one.
\end{remark}
\begin{remark}
    Pointwise limit of continuous functions (or Riemann integrals) can be
    discontinuous (or not Riemann integrable).
    For example, if we number all of the rational numbers, and construct 
    continuous functions that raise to the rational numbers and then 
    go down again for the first $n$ rational numbers for every $n$, 
    then the limit of such functions is the Dirichlet function.

    Despite that, the following proposition holds true true:
\end{remark}
\begin{proposition}
    $f_n : E \to \overline{\mathbb{R}}$ --- measurable.
    If $f_n \to f$ pointwise almost everywhere on $E$, then $f$ is measurable.
\end{proposition}
\begin{proof}
    Without the loss of generality, the convergence is pointwise.
    (Because if we prove it for a subset of $E$ of the same measure, then
    returning the set of points of measure zero to our domain of definition
    will not change the measurability).

    Let's consider $f^{-1}\bigl([-\infty, c]\bigr) = \{ x \mid f(x) < c\}$.
    \[
        f(x) < c \Longleftrightarrow \exists n, k \in \mathbb{N}:
        \forall j > k:
        f_j(x) < c - \frac{1}{n}
    \]
    (Note that this $\frac{1}{n}$ part is important, because if we omitted that,
    it could happen that $f_j(x)$ converge to $c$, but are always strictly less than $c$.)
    Therefore,
    \[
        f^{-1}\bigl([-\infty, c)\bigr) = 
        \bigcup_{k, n}
        \bigcap_{j=k+1}^\infty f_j^{-1}\biggl(\Bigl[-\infty, c - \frac{1}{n}\Bigr)\biggr)
    \]
    which is measurable because countable unions and intersections keep measurability.
\end{proof}

\subsection{Simple functions}
\begin{definition}
    \label{def:simpleFunction}
    Let $E \in \mathcal{M}$.
    A function $\varphi : E \to \mathbb{R}$ is called a \textit{simple}, if it is
    measurable and takes only finitely many values.
\end{definition}
\begin{property}
    A simple function can be represented in the following way
    (also called the canonical representation):
    \[
        \varphi = \sum_{k=1}^n c_k \cdot \chi_{E_k}
        \text{, where } E_k = \varphi^{-1}(c_k)
        %\text{ and } \forall i \ne j: E_i \cap E_j = \emptyset
    \]
    (Here $c_k$ are all the values of $\varphi$, $\chi$ is the characteristic function.)
\end{property}
\begin{remark}
    We could also represent a simple function in the following way:
    \[ \varphi = \sum_{k=1}^n c_n \chi_{E_k} \]
    Here $E_k$ may overlap. But we can easily derive
    a canonical representation. Say, $E_i \cap E_j \ne \emptyset$.
    Then
    \[ \varphi = c_i \cdot \chi_{E_i \setminus E_j} + c_j \cdot \chi_{E_j \setminus E_i} +
        (c_i + c_j) \cdot \chi_{E_i \cap E_j} + \dots \]
\end{remark}
\begin{lemma}[Simple approximation lemma]
    \label{lem:simpleApproxLemma}
    $f : E \to \mathbb{R}$ --- measurable and bounded
    ($\exists M: \abs{f} < m$).
    Then for every $\varepsilon > 0$ there exist 
    simple functions $\varphi_\varepsilon, \psi_\varepsilon : E \to \mathbb{R}$,
    such that $\varphi_\varepsilon \le f \le \psi_\varepsilon$ and
    $0 \le \psi_\varepsilon - \varphi_\varepsilon < \varepsilon$ on $E$.
\end{lemma}
\begin{proof}
    Let's split the range of function $f$ into equal intervals
    $I_1, \dots, I_n$ of length no more than $\varepsilon$.
    Now, for every $x \in E$, let's round $f(x)$ \textit{down} to the closest interval end point and
    call the resulting function $\varphi_\varepsilon(x)$, and let's round $f(x)$ 
    \textit{up} to the closest interval end point and call the resulting function $\psi_\varepsilon(x)$.

    Then indeed we'll have $\varphi_\varepsilon \le f \le \psi_\varepsilon$ (since
    we rounded the function down or up, respectively), and
    $0 \le \psi_\varepsilon - \varphi_\varepsilon < \varepsilon$ as 
    the rounded points will end in the same interval for every $x \in E$.
\end{proof}
