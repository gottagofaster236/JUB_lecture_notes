\begin{theorem}[Simple approximation theorem]
    $f: E \to \overline{\mathbb{R}}$ is measurable if and only if 
    there exists a sequence of simple functions $\varphi_n : E \to \mathbb{R}$,
    such that $\varphi_n \to f$ pointwise on $E$, and $\abs{\varphi_n} \le \abs{f}$
    on $E$ for all $n$.

    (If $f \ge 0$, then $\{\varphi_n\}$ can be chosen to be increasing.)
\end{theorem}
\begin{remark}
    In the simple approximation lemma, functions 
    $\varphi_\varepsilon$ and $\psi_\varepsilon$ sandwich the function $f$,
    and\\ $0 \le \psi_\varepsilon - \varphi_\varepsilon < \varepsilon$, therefore, 
    at every point they both differ from $f$ by no more than $\varepsilon$.
    If we could choose a sequence of $\varepsilon$'s that converges to 0,
    we would get uniform convergence.

    Unfortunately, we can't apply the lemma directly, because we needed the function $f$ to be bounded.
    That's why we're not getting uniform convergence.
\end{remark}
\begin{proof}
    \begin{enumerate}
        \item {
            $\impliedby$. We have a sequence of simple (and thus measurable) functions $\{\varphi_n\}$,
            therefore, their limit is also measurable.
        }
        \item {
            $\implies$.
            \begin{enumerate}[label={(\arabic*)}]
                \item {
                    Let's express $f = f^+ - f^-$, where $f^+ = \max\{f, 0\}$ and $f^- = \min\{f, 0\}$.
                    Let's prove the theorem for both $f^+$ and $f^-$.
                    Without the loss of generality, assume that $f \ge 0$.
                }
                \item {
                    To use the simple approximation lemma, we have to make the function bounded.
                    Define $f_n \coloneqq \min\{f, 2^n\}$.
                    Take $\varphi_n$ as the lower simple approximation of $f_n$,
                    according to the simple approximation lemma, with $\varepsilon = \frac{1}{2^n}$.

                    Then $\varphi_n \to f$ pointwise. Moreover, $\{\varphi_n\}$ 
                    are increasing (because we bound it with $2^n$, and because when we 
                    go from $n$ to $n + 1$, every interval in the proof of SAL is split into two).
                }
                \item {
                    Do the same for $f^-$ and add the sequences for $f^+$ and $f^-$ together.
                    Since $\abs{f} = \max\{\abs{f^+}, \abs{f^-}\}$, the condition
                    of $\abs{\varphi_n} \le \abs{f}$ will hold true. 
                }
            \end{enumerate}
        }
    \end{enumerate}
\end{proof}

\subsection{Egoroff, Lusin, Littlewood's three principles}
Informal statement:
\begin{enumerate}
    \item {
        Every measurable set is \textit{nearly} (not a mathematical term) a finite union of intervals.
    }
    \item {
        Every measurable function is nearly continuous.
    }
    \item {
        Every pointwise convergent sequence of measurable functions is nearly uniform convergent.
    }
\end{enumerate}
\begin{theorem}[Egoroff's theorem]
    Assume that $m(E) < \infty$, 
    $f_n : E \to \overline{\mathbb{R}}$ --- measurable, $f_n \to f$
    pointwise almost everywhere on $E$, where $f : E \to \mathbb{R}$.
    Then for any $\varepsilon > 0$ there exists a closed subset
    $F \subset E$, such $m(E \setminus F) < \varepsilon$ and $f_n \to f$
    uniformly on $f$.
\end{theorem}
\begin{remark}
    It is crucial that $f$ is from $E$ to $\mathbb{R}$, not $\mathbb{\overline{R}}$,
    otherwise we wouldn't get uniform convergence.
\end{remark}
\begin{proof}
    $\abs{f} < \infty \implies \abs{f - f_k}$ is defined for all $x \in E$
    (starting from some $k$).

    Let's also throw out the subset of measure $E$, on which $f_n \not\to f$.

    \begin{enumerate}
        \item {
            Fix $\eta > 0$.
            \[ E_n \coloneqq \{ x \in E \mid \abs{f(x) - f_k(x)} < \eta,\ \forall k \ge n \} \]
            Each set $E_n$ is measurable, because $\abs{f(x) - f_k(x)}$ is measurable,
            and $E_n$ is the countable intersection of preimages of $[0, \eta)$ for all $k$.
            We can also note that $\{E_n\}$ is ascending.
            
            Also, $ \cup_{n=1}^\infty E_n = E $.
            That's follows from pointwise convergence. If we take any point from the set $E$,
            eventually it will be contained in some $E_n$, because for a given point $x$
            the difference $\abs{f(x) - f_k(x)}$ has to converge to 0. 
            Therefore,
            \[
                \lim_{n \to \infty} m(E_n) = m(E)
            \]
            Then there exists $N \in \mathbb{N}$, such that 
            $m(E \setminus E_n) < \varepsilon$ for every $\varepsilon > 0$>
        }
        \item {
            Now let's take such $N$ for $\eta_n = \frac{1}{n}$,
            $\varepsilon_n = \frac{\varepsilon}{2^{n + 1}}$.

            Take $E_n = E_N$, where $E_N$ is from step 1
            for $\eta = \eta_n$ and $\varepsilon = \varepsilon_n$.

            Take $A = \cap_{n=1}^\infty E_n$. Then, by convergence, 
            $f_n \to f$ uniformly on $A$. On the other hand, 
            $A$ is measurable as a countable union of measurable sets.

            Let's estimate the measure of $A$.
            \[ m(E \setminus A) = m\Bigl(
                \bigcup_{k=1}^\infty (E \setminus E_n)
            \Bigr) \le
            \sum_{n=1}^\infty m(E \setminus E_n) \le
            \sum_{n=1}^\infty \frac{\varepsilon}{2^{n+1}} = \frac{\varepsilon}{2}
            \]
            But $A$ may be non-closed, so we can't just assign $F$ to $A$.
            Therefore, we have to approximate $A$ by closed sets.
        }
        \item {
            Take $F \subset A$, so that $F$ is closed and
            $m(A \subset F) < \frac{\varepsilon}{2}$.
            We \hyperref[the:lebesgueMeasurableConditions]{have proved} that such an $F$ exists if $A$ is measurable.
        }
    \end{enumerate}
\end{proof}