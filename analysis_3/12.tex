\begin{theorem}[Lusin's theorem]
    $f : E \to \mathbb{R}$ --- measurable. Then for every
    $\varepsilon > 0$ there exists a continuous function
    $g : \mathbb{R} \to \mathbb{R}$ and a closed set $F \subset E$, 
    such that $f = g$ on $F$ and $m(E \setminus F) < \varepsilon$.
\end{theorem}
\begin{proof}
    \textit{Main idea}: uniform limit of continuous functions is continuous,
    then apply Egoroff's theorem.

    The proof is gonna be split into two parts:
    \begin{enumerate}
        \item {
            Find a closed $F$ on which $f$ is continuous (in induced topology).
        }
        \item {
            Extend $f|_F$ to a continuous function $g$.
            We have briefly discussed how to do this:
            the complement of a closed set $F$ is an open set, 
            and thus a disjoint union of open intervals. On every such
            open interval we can connect the graph of $f$
            with a straight line. There's some work to be done, because
            there can be an infinite number of such intervals converging a point,
            but we'll leave it as an exercise.
        }
    \end{enumerate}

    Proof of Part 1:
    \begin{enumerate}
        \item {
            Assume $m(E) < \infty$. (Let's leave $m(E) = \infty$
            as an exercise).
            By simple approximation lemma, there exists a sequence of 
            simple functions $f_n : E \to \mathbb{R}$, such that
            $f_n \to f$ pointwise.
        }
        \item {
            $\forall n$ there exists a closed subset $F_n \subset F$,
            such that $f_n$ is continuous on $F_n$ and
            $m(E \setminus F_n) < \frac{\varepsilon}{2^{n+1}}$.
            
            Proof:
            \[ f_n = \sum_{j=1}^{k_n} c_j \cdot \chi_{E_j} \text{ --- canonical representation} \]
            Each of the sets $E_j$ 
            \hyperref[the:lebesgueMeasurableConditions]{can be approximated}
            by a closed set $G_j$ from the inside (they are measurable from
            \hyperref[def:simpleFunction]{the definition of a simple function}).

            Put $F_n \coloneqq \cup G_j$. It's a finite union, therefore $F_n$ is closed.

            $f_n$ is constant on each of $G_j$, as $G_j \subset E_j$.
            A preimage of an open set for $f$ was a union of several $E_j$'s.
            If we now restrict $f$ onto $\cup G_j$, the preimage of an open set
            will be a union of several $G_j$'s. What we have to prove that 
            $G_j$'s are open in the induced topogy on $F_n$ --- that's 
            true because $G_j$'s are closed in $\mathbb{R}$ and disjoint.
        }
        \item {
            By Egoroff's theorem, there exists a closed $F_0 \subset E$, 
            such that $m(E \setminus F_0) < \frac{\varepsilon}{2}$ and 
            $f_n \to f$ uniformly.
            Now take $F \coloneqq \cap_{n=0}^\infty F_n$. The intersection of closed sets 
            is closed. 
            By construction, $f_n$ is continuous on $F$, as $F_n \subset F$.
            Since $f_n \to f$ uniformly on $F$ (as $F_0 \subset F$),
            $f$ is continuous on $F$.
            Let's now estimate the measure of $E \setminus F$:
            \[
                m(E \setminus F) < \frac{\varepsilon}{2} + 
                \sum_{n=1}^\infty \frac{\varepsilon}{2^{n+1}} = \varepsilon
            \]
        }
    \end{enumerate}
\end{proof}
\begin{remark}
    If $m(E) = \infty$, we can split $\mathbb{R}$ into intervals with finite length.
    Then use the finite case we've just proved for each of those intervals,
    but choose $\varepsilon_k = \frac{\varepsilon}{2^{k + 1}}$ so that they all sum 
    up to $\varepsilon$ at the end.
\end{remark}

\section{Lebesgue integral}
\begin{definition}
    Let $\psi : E \to \mathbb{R}$ be simple, $m(E) < \infty$. Let
    \[ \psi = \sum_{n=1}^k a_n \cdot \chi_{E_n} \]
    be the canonical representation of $\psi$. Then define
    \[ \int_E \psi \coloneqq \sum_{n=1}^k a_n \cdot m(E_n) \]
\end{definition}
\begin{remark}
    Notation: if we want to emphasize the measure, we can write
    \[ \int_E \psi = \int_E \psi \,dm = \int_a^b \psi \,dm \text{ (when $E=[a, b]$)} \]
\end{remark}
\begin{remark}
    One can show that if it's not the canonical representation,
    we can still compute the integral by the same formula and get the same result.
\end{remark}

\begin{definition}
    Let $f : E \to \mathbb{R}$ be bounded, $m(E) < \infty$.
    ($f$ is not necessarily measurable).

    The \textit{lower Lebesgue integral} is by definition
    \[ \sup\Bigl\{ \int_E \varphi \mathrel{\Big\vert} \text{ $\varphi$ is simple and $\varphi \le f$ on $E$} \Bigr\} \]
    The \textit{upper Lebesgue integral} is by definition
    \[ \inf\Bigl\{ \int_E \psi \mathrel{\Big\vert} \text{ $\varphi$ is simple and $\varphi \ge f$ on $E$} \Bigr\} \]
    $f$ is \textit{Lebesgue integrable} if the lower Lebesgue integral
    is equal to the upper Lebesgue integral. Then
    \[ \int_E f \coloneqq \text{lower Lebesgue integral} = \text{upper Lebesgue integral} \]
\end{definition}

\begin{theorem}
    $f : [a, b] \to \mathbb{R}$ is bounded and $[a, b]$ is bounded.
    If $f$ is Riemann integrable over $[a, b]$, then 
    $f$ is Lebesgue integrable and both integrals will match.
\end{theorem}
\begin{proof}
    Upper and lower Riemann sums are obtained by simple functions.
\end{proof}
\begin{theorem}
    If $f : E \to \mathbb{R}$ is bounded and \textit{measurable}, 
    $m(E) < \infty$, then $f$ is Lebesgue integrable on $E$.
\end{theorem}
\begin{proof}
    By the simple approximation lemma, for every $n \in \mathbb{N}$
    there exist simple functions $\varphi_n$, $\phi_n$, such that
    $\varphi_n \le f \le \psi_n$ on $E$ and
    $\psi_n - \varphi_n \le \frac{1}{n}$ on $E$.
    \[
        \forall n: 
        (\text{upper Lebesgue integral} - \text{lower Lebesgue integral}) \le
        \int_E \psi_n - \int_E \varphi_n = \int_E (\psi_n - \varphi_n) \le
        \frac{1}{n} m(E) \to 0
    \]
    Here we've used the fact that the difference of two integrals
    is the integral of the difference (which we haven't yet proved).
    However, since we're dealing with simple functions,
    we can prove it by simply rearranging the finite sums.
\end{proof}

Now here's some definitions for the improper case,
i.e. infinite integration limits or functions that converge to infinity.

\begin{definition*}
    $f : E \to \overline{\mathbb{R}}$ is measurable.
    Its support is by definition
    \[ \supp f \coloneqq \{ x \in E \mid f(x) \ne 0 \} \]
    $f$ is of finite support of $m(\supp f) < \infty$.
\end{definition*}
\begin{definition}
    $f : E \to \overline{\mathbb{R}}$ is measurable, $f \ge 0$. Then
    \[
        \int_E f \coloneqq \sup \Bigl\{
            \int_E h \mathrel{\Big\vert}
            \text{$h$ is bounded, measurable,}\
            m(\supp h) < \infty,\ 0 \le h \le f \text{ on } E
        \Bigr\} 
    \]
    $f$ is Lebesgue integrable over $E$, if $\int_E f < \infty$.
\end{definition}
\begin{definition}
    A measurable function $f : E \to \overline{\mathbb{R}}$
    is Lebesgue measurable over $E$ if 
    $f^+ = \max\{f, 0\}$ and $f^- = \max\{-f, 0\}$ are Lebesgue
    integrable over $E$.
    Then
    \[ \int_E f = \int_E f^+ - \int_E f^- \]
\end{definition}
