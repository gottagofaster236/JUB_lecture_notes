\begin{definition}
    A function $f: E \to \overline{\mathbb{R}}$
    is \text{integrable} if both $\int_E f^+$ and $\int_E f^-$ are finite.
\end{definition}
\begin{remark}
    If $\int_E f^+$ and $\int_E f^-$ are both infinite, we can't subtract them in the previous
    definition. Therefore, it is not enough to just say that $\int_E f$ has to be finite.
\end{remark}

\begin{theorem}
    \label{the:integralLinearityIntegrable}
    $f, g: E \to \mathbb{R}$ are measurable and integrable. Then 
    for any $\alpha, \beta \in \mathbb{R}$ the linear combination
    $\alpha f + \beta g$ is integrable, and 
    \[ \int_E (\alpha f + \beta g) = \alpha \int_E f + \beta \int_E g \]
\end{theorem}
\begin{theorem}
    \label{the:integralLinearityPositive}
    $f, g: E \to \mathbb{R}$ are measurable and non-negative. Then for any 
    $\alpha, \beta \ge 0$:
    \[ \int_E (\alpha f + \beta g) = \alpha \int_E f + \beta \int_E g \]
\end{theorem}
\begin{proof}
    The proof of these two theorems is technical, we are not gonna prove them here.
    You can read them in the textbook.
    Checking the linearity of simple functions is easy. But the moment 
    we proceed to supremums and infimums it becomes more difficult.
\end{proof}
\begin{remark}
    Why do we need two theorems instead of one? Notice that 
    Theorem~\ref{the:integralLinearityPositive} doesn't 
    require integrability. If $f$ and $g$ are non-negative, then 
    their integrals are non-negative as well, therefore, we can add the two 
    integrals and not worry about cases like $+\infty + (-\infty)$.

    In Theorem~\ref{the:integralLinearityIntegrable}, we require integrability. Therefore, the integrals
    of $f$ and $g$ are both finite, therefore, the right-hand side is well-defined.
    But in $\int_E (\alpha f + \beta g)$ we could indeed have 
    $+\infty + (-\infty)$. But that's not a problem: we will prove later,
    that if a function is integrable, then it's infinite only on a subset
    of measure 0.
\end{remark}

\subsection{Basic properties of Lebesgue integral}
\begin{theorem}[Monotonicity]
    Assume $f, g: E \to \overline{\mathbb{R}}$ are measurable and
    either both integrable or both non-negative. Then 
    if $f \le g$ on $E$, then 
    \[ \int_E f \le \int_E g \]
\end{theorem}
\begin{proof}
    We can use either Theorem~\ref{the:integralLinearityIntegrable} or 
    Theorem~\ref{the:integralLinearityPositive}.
    If $g = f + h$, then $g \ge f \implies h \ge 0 \implies \int_E h \ge 0$.
    But $\int_E g = \int_E f + \int_E h$ by one of those theorems, therefore,
    $\int_E g \ge \int_E f$.
\end{proof}

\begin{proposition}
    $f : E \to \overline{\mathbb{R}}$ is measurable. Then
    $f$ is integrable $\iff$ $\abs{f}$ is integrable.
\end{proposition}
\begin{proof}
    $\abs{f} = f^+ + f^-$.
    \begin{enumerate}
        \item {
            $\implies$.
            If $f$ is integrable, then $\int f^+,\ \int f^- < \infty$ by definition.
            Then
            \[ \int \abs{f} = \int f^+ + \int f^- < \infty \]
        }
        \item {
            $\impliedby$.
            $0 \le f^+ \le \abs{f}$ and $0 \le f^- \le \abs{f}$.
            If $\abs{f}$ is integrable, then, by monotonicity, $f^+$ and $f^-$
            are integrable.
        }
    \end{enumerate}
\end{proof}

\begin{proposition}
    \label{prop:boundedIntegrable}
    $f : E \to \overline{\mathbb{R}}$, $g : E \to \overline{\mathbb{R}}$ --- integrable and
    $g \ge 0$. If $\abs{f} \le g$ on $E$, then $f$ is integrable.
\end{proposition}
\begin{proof}
    If $g$ is integrable, then by monotonicity $\abs{f}$ is integrable,
    then by previous proposition $f$ is integrable.
\end{proof}

\begin{theorem}
    \label{the:integralOnUnion}
    $f: E \to \overline{\mathbb{R}}$ is measurable and either integrable
    or non-negative. If $A, B \subset E$, $A \cap B = \emptyset$;
    $A, B \in \mathcal{M}$, then
    \[ \int_{A \cup B} f = \int_A f + \int_B f \]
\end{theorem}
\begin{proof}
    \[ f\big|_{A \cup B} = f \cdot \chi_A + f \cdot \chi_B = h + g \]
    If $f$ is integrable, then $h$ and $g$ are integrable by previous proposition,
    because $\abs{h} \le \abs{f}$ and $\abs{g} \le \abs{f}$.
    Now apply Theorem~\ref{the:integralLinearityIntegrable}.

    If $f$ is non-negative, then $h, g \ge 0$, apply Theorem~\ref{the:integralLinearityPositive}.
\end{proof}

\begin{proposition}
    If $f : E \to \overline{\mathbb{R}}$ is integrable, then
    \[ \abs{\int_E f} \le \int_E \abs{f} \]
\end{proposition}
\begin{proof}
    \[ \abs{\int_E f} = \abs{\int_E f^+ - \int_E f^-} \le \int_E f^+ + \int_E f^- = \int_E \abs{f} \]
\end{proof}
\begin{theorem}[Chebyshev's inequality]
    \label{the:cheb}
    $f : E \to \overline{\mathbb{R}}$ --- measurable, $f \ge 0$. Then $\forall \lambda > 0$:
    \[ m(E_\lambda) = m \{ x \in E \mid f(x) \ge \lambda \} \le \frac{1}{\lambda} \int_E f \]
\end{theorem}
\begin{proof}
    Put $h \coloneqq \lambda \cdot \chi_{E_\lambda}$. Since
    $h$ is a simple function,
    \[ \int_{E_\lambda} h = \lambda \cdot m(E_\lambda) \]
    Therefore, it is enough to prove that
    \[ \int_{E_\lambda} h  \le \int_E f \]
    But $h \le f \big|_{E_\lambda}$,
    because on $E_\lambda$ we have $h = \lambda$ and $f \ge \lambda$. Therefore, by monotonicity, 
    \[ \int_{E_\lambda} h \le \int_{E_\lambda} f \le \int_{E} f \]
\end{proof}

\subsubsection*{Applications of the Chebyshev's inequality:}
\begin{proposition}
    $f : E \to \overline{\mathbb{R}}$ is measurable, $f \ge 0$.
    Then \[ \int_E f = 0 \iff f = 0 \text{ almost everywhere on } \mathbb{R} \]
\end{proposition}
\begin{proof}\mbox{}
    \begin{enumerate}
        \item {
            $\implies$.

            If $\int_E f = 0$, then by Chebyshev's inequality,
            \[ m(E_n) = m\Bigl\{x \in E \mathrel{\Big\vert} f(x) \ge \frac{1}{n} \Bigr\}  \le n \int_E f = 0 \]
            The sequence $\{E_n\}$ is ascending. Since the measure of each one of them
            is equal to zero, then, by \hyperref[the:continuityOfMeasure]{continuity of measure},
            $m(\cup_{n=1}^\infty E_n) = m\{x \in E \mid f(x) > 0 \} = 0$.
        }
        \item {
            $\impliedby$. Follows from Definition~\ref{def:definitionThree}.
        }
    \end{enumerate}
\end{proof}

\begin{proposition}
    If $f: E \to \overline{\mathbb{R}}$ is integrable, then $f$ is finite
    almost everywhere on $E$.
\end{proposition}
\begin{proof}
    Without the loss of generality, assume that $f \ge 0$
    (otherwise, prove separately for $f^+$ and $f^-$).
    \[ 
        m\{x \in E \mid f(x) = \infty \} \le m\{x \in E \mid f(x) \ge n \} 
        \overset{\text{Chebyshev's inequality}}{\le} \frac{1}{n} \int_E f 
        \underset{n \to \infty}{\longrightarrow} 0
    \]
    Here we've used that $\int_E f$ is finite, because $f$ is integrable.
\end{proof}
