\subsection{Convergence theorems}
\begin{proposition}
    \label{prop:integralUniformLimit}
    $f_n : E \to \overline{\mathbb{R}}$ --- measurable, integrable, $m(E) < \infty$. If $f_n \to f$ uniformly and
    $f$ is bounded on $E$, then 
    \[ \lim_{n \to \infty} \int_E f_n = \int_E f \] 
\end{proposition}
\begin{proof}
    $f$ is measurable (as a pointwise limit). Therefore, $\int_E f$ exists.
    For any $\varepsilon > 0$, choose $N > 0$, such that
    \[
        \abs{f - f_n} < \frac{\varepsilon}{m(E)}, \quad \forall n \ge N
    \]
    Then
    \[
        \abs{\int_E f - \int_E f_n} = \abs{\int_E(f - f_n)} \le
        \int_E \abs{f - f_n} < \frac{\varepsilon}{m(E)} \cdot m(E) = \varepsilon
    \]
\end{proof}

\begin{remark}
    This proposition does not hold for pointwise convergence.
\end{remark}
\begin{example}[1]
    $E = (0, 1],\ f_n \coloneqq n \cdot \chi_{(0, 1/n)}$.
    Then $f_n \to 0$ pointwise. But:
    \[ \int_E f_n = 1 \quad \int_E f = 0 \]
\end{example}
\begin{example}[2]
    $E = \mathbb{R},\ f_n = \chi_{[n, n + 1]}$.
    If we choose an $x$, then for any $n > x$
    this interval will be to the right of $x$, and thus $f_n(x) = 0$.
    Therefore, $f_n \to 0$ pointwise. But:
    \[ \int_E f_n = 1 \quad \int_E f = 0 \]
\end{example}

\begin{theorem}[Bounded convergence theorem]
    $f_n : E \to \overline{\mathbb{R}}$ --- measurable, $m(E) < \infty$.
    Assume that there exists an $M > 0$, such that $\abs{f_n} < M$ on $E$ for all $n$.
    Then if $f_n \to f$ pointwise on $E$, then 
    \[ \lim_{n \to \infty} \int_E f_n = \int_E f \]
\end{theorem}
\begin{remark}
    If we look at the two examples stated earlier as a movie about a function,
    with frames $f_n$, then we can see how the ``mass'' of a function \textit{(not formal)} is moving.
    In the first example, the ``mass'' of the function was escaping vertically. Here
    it's prevented by $\abs{f_n} < M$. In the second example, the ``mass''
    was escaping horizontally. Here it's prevented by $m(E) < \infty$.
\end{remark}
\begin{proof}
    We're gonna use Egoroff's theorem.
    \begin{enumerate}
        \item {
            $f$ is measurable (as a pointwise limit). 
            $\abs{f} \le M$.
        }
        \item {
            For any $\varepsilon > 0$, take $A \subset E$, such that
            $m(E \setminus A) < \frac{\varepsilon}{4M}$ and $f_n \to f$
            uniformly on $A$ (by Egoroff's theorem).
            Then, 
            \[
                \abs{\int_E f_n - \int_E f} \le
                \abs{\int_A f_n - \int_A f} +
                \abs{\int_{E \setminus A} f_n - \int_{E \setminus A} f} = (*)
            \]
            Since $f_n \to f$ uniformly on $A$, we have (by Proposition~\ref{prop:integralUniformLimit})
            that $\abs{\int_A f_n - \int_A f} \to 0$ as $n \to \infty$.
            Now,
            \[
                \abs{\int_{E \setminus A} f_n - \int_{E \setminus A} f} \le
                2M \cdot m(E \setminus A) < \frac{\varepsilon}{2}
            \]
            Since we can make the first modulus be smaller than 
            $\frac{\varepsilon}{2}$ by choosing a sufficiently large $n$, 
            we get that
            \[ (*) \le \frac{\varepsilon}{2} + \frac{\varepsilon}{2} = \varepsilon \]
        }
    \end{enumerate}
\end{proof}

\begin{lemma}[Fatou's lemma]
    \label{lem:fatou}
    $f_n : E \to \overline{\mathbb{R}}$ --- measurable, $f_n \ge 0$.
    If $f_n \to f$ pointwise almost everywhere on $E$, then
    \[
        \int f \le \liminf_{n \to \infty} \int_E f_n 
    \]
\end{lemma}
\begin{remark}
    The sequence of integrals may not have a limit, therefore, we have to use
    $\liminf$ in the statement. By definition,
    \[ \liminf_{n \to \infty} a_n = \lim_{k \to \infty} \inf_{n \ge k} a_n \]
    The since $\inf_{n \ge k} a_n$ is a non-decreasing sequence, the limit
    (possibly infinite) always exists.
\end{remark}
\begin{remark}
    If we think in terms of our escape of the mass analogy, the theorem states that mass can escape,
    but can't enter the limit.
\end{remark}
\begin{proof}
    Exclude a set of measure zero (by Egoroff's theorem), now the convergence is pointwise.
    $f \ge 0$ and $f$ is measurable.

    Recall Definition~\ref{def:definitionThree}.
    Take $h : E \to \overline{\mathbb{R}}$ to be any bounded measurable function with 
    $m(\supp h) < \infty$ and $0 \le h \le f$ on $E$, like in that definition.

    Let $M$ be such that $h \le M$ on $E$ (since $h$ is bounded) and $E_0 \coloneqq \supp h$.
    Define $h_n : E \to \mathbb{R}$ by $h_n \coloneqq \min\{h, f_n\}$. Then:
    \begin{enumerate}
        \item {
            $h_n$ is measurable, because both $h$ and $f_n$ are measurable by their
            respective definitions, and $h_n$ is the minimum of two measurable functions.
        }
        \item {
            $0 \le h_n \le M$, because $h \le M$ on $E$.
        }
        \item {
            $\supp h_n \subset E_0$, as $h_n \le h$.
        }
        \item {
            $h_n \to h$ pointwise on $E$. That's because $f_n$ converges to $f$ pointwise
            and $h \le f$. Therefore, at every point, starting from some $n$, 
            Thus, $h$ will be either less than $f_n$ or arbitrarily close to it, if
            $h = f$ at that point. Since $h_n = \min\{f, h\}$ and 
            $\min\{f, h\} \to h$, we get that $h_n \to h$.
        }
    \end{enumerate}
    By Bounded convergence theorem:
    \[ 
        \lim_{n \to \infty} \int_E h_n 
        \overset{\supp h_n \subset E_0}{=} \lim_{n \to \infty} \int_{E_0} h_n
        \overset{\text{BCT}}{=}
        \int_{E_0} h \overset{\supp h = E_0}{=} \int_E h
    \]
    On the other hand,
    \[
        h_n \le f_n \implies \int_E f_n \ge \int_E h_n \implies
        \int_E h \le \liminf_{n \to \infty} \int_E f_n
    \]
    Since $h$ is arbitrary, by Definition~\ref{def:definitionThree},
    \[
        \int_E f \le \liminf_{n \to \infty} \int_E f_n
    \]
\end{proof}

\begin{theorem}[Monotone convergence theorem]
    $f_n : E \to \overline{\mathbb{R}}$, $f_n \ge 0$, measurable.
    The sequence $\{f_n\}$ is increasing on $E$
    ($f_n \ge f_k$ for $n > k$).
    If $f_n \to f$ pointwise almost everywhere on $E$, then 
    \[ \lim_{n \to \infty} \int_E f_n = \int_E f \]
\end{theorem}
\begin{proof}
    \mbox{}
    \begin{enumerate}
        \item {
            From Fatou's lemma,
            \[ \int_E f \le \liminf \int_E f_n \]
        }
        \item {
            \[ f_n \nearrow \implies \int_E f \ge \limsup \int_E f_n \implies
            \text{all three are equal} \]
        }
    \end{enumerate}
\end{proof}

\begin{theorem}[Lebesgue's dominated convergence theorem]
    \label{the:domConv}
    Let $f_n: E \to \overline{\mathbb{R}}$ --- measurable. Let
    $g : E \to \overline{\mathbb{R}}$ be integrable over $E$, $g \ge 0$.
    Assume that $\abs{f_n} \le g$ for all $n$.

    If $f_n \to f$ pointwise almost everywhere on $E$, then $f$ is integrable
    on $E$ and
    \[
        \lim_{n \to \infty} \int_E f_n = \int_E f
    \]
\end{theorem}
\begin{remark}
    Bounded convergence theorem is a special case with
    \[
        g(x) = \begin{cases}
            M,& x \in E\\
            0,& x \not\in E
        \end{cases}
    \]
\end{remark}
\begin{proof}
    The idea of the proof is to apply Fatou's lemma to $g + f$ and $g - f$.
    \begin{enumerate}
        \item {
            Since $\abs{f_n} \le g$, we have $\abs{f} \le g$ almost everywhere on $E$.
            Therefore, $f$ is integrable as we proved \hyperref[prop:boundedIntegrable]{earlier}.
        }
        \item {
            Remove a subset of measure 0 so that $f$ and each $f_n$ are finite on $E$.
            That's possible, because each of the functions $f$ and $f_n$ has an infinite value on a subset of measure 0,
            and the countable union of subsets of measure 0 also has measure 0.
            Now $g \pm f$ and $g \pm f_n$ are well-defined.
        }
        \item {
            Apply Fatou's lemma:
            \begin{enumerate}
                \item {
                    $f_n \to f$, and $g - f_n$ are non-negative due to the dominance by $g$.
                    \[ 
                        \int_E (g - f) \le \liminf \int_E (g - f_n) =
                        \int_E g- \limsup \int_E f_n
                    \]
                    Now we can cancel out $\int_E g$ on both sides, since it's finite. We get that
                    \[
                        \int_E f \ge \limsup \int_E f_N
                    \]
                }
                \item {
                    Apply Fatou's lemma to the sequence of $g + f_n$, which is also 
                    non-negative due to the dominance by $g$:
                    \[
                        \int_E (g + f) \le \liminf\int_E(g + f_n) =
                        \int_E g + \liminf \int_E f_n \implies
                        \int_E f \le \liminf \int_E f_n
                    \]
                }
            \end{enumerate}
            \[
                \text{(a)} + \text{(b)} \implies \int_E f = \lim_{n \to \infty} \int_E f_n
            \]
        }
    \end{enumerate}
\end{proof}

\begin{theorem}[Countable additivity of integration]$ $\\
    $f : E \to \overline{\mathbb{R}}$ --- integrable.
    $\{E_n\}^\infty$ are disjoint measurable subsets of $E$,
    $E = \cup_{1}^\infty E_n$.
    Then
    \[ \int_E f = \sum_{n=1}^\infty \int_{E_n} f \]
\end{theorem}
\begin{proof}
    Let
    $\chi_n \coloneqq \chi_{\cup_1^n E_k},\ f_n \coloneqq f \cdot \chi_n$.
    Then $f_n$ are measurable and $\abs{f_n} \le \abs{f}$ on $E$
    and $f_n \to f$ pointwise on $E$.
    Hence by Lebesgue's dominance convergence theorem (with $g = f$),
    \[
        \int_E f = \lim_{n \to \infty} \int_E f_n \overset{(*)}{=} \sum_1^\infty \int_{E_n} f
    \]
    Here (*) holds true, because the for a finite sum this equality is true by
    applying induction to \hyperref[the:integralOnUnion]{this theorem}, and 
    then we can take the limits.
\end{proof}

\begin{theorem}[Continuity of integration]
    $f : E \to \overline{\mathbb{R}}$ --- integrable.
    \begin{enumerate}
        \item {
            If $\{E_n\}_1^\infty$ --- ascending, each $E_n$ is measurable, then
            \[
                \int\limits_{\cup_1^\infty E_n} f = \lim_{n \to \infty} \int_{E_n} f
            \]
        }
        \item {
            If $\{E_n\}_1^\infty$ --- descending, each $E_n$ is measurable, then
            \[
                \int\limits_{\cap_1^\infty E_n} f = \lim_{n \to \infty} \int_{E_n} f
            \]
        }
    \end{enumerate}
\end{theorem}
\begin{proof}
    We're not gonna give the full proof, just the proof idea.
    We can use the dominated convergence theorem.
    
    In the first case, define $f_n \coloneqq f \cdot \chi_{E_n}$.
    Since it's an ascending sequence, we'll have $f_n \to f$ pointwise on 
    $\cup_1^\infty E_n$.
    Now apply dominated convergence theorem with $g = f$ on $\cup_1^\infty E_n$.

    The second case is analogous, but we'll have to look at the complement.
\end{proof}