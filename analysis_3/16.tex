\subsection{Useful inequalities}
\begin{definition}[Conjugate]
    For any $p \in (1, +\infty)$, we will call $q \in (1, +\infty)$
    its \textit{conjugate}, if 
    \[ \frac{1}{p} + \frac{1}{q} = 1 \]
    We will denote it as $q = \overline{p}$.
\end{definition}
\begin{remark}
    This is not the same conjugate as in complex analysis.
\end{remark}
\begin{remark}
    It's natural to say that the conjugate of 1 is $+\infty$, and vice versa.
\end{remark}

\begin{theorem}[Young's inequality]
    For all $a, b > 0,\ 1 < p < \infty$ and $q = \overline{p}$, we have
    \[
        ab \le \frac{a^p}{p} + \frac{b^q}{q}
    \]
\end{theorem}
\begin{proof}
    Consider $g(x)$:
    \begin{align*}
        &
        g(x) = \frac{x^p}{p} + \frac{1}{q} - x
        \\&
        g(1) = \frac{1}{p} + \frac{1}{q} - 1 = 0
        \\&
        g'(x) = x^{p-1} - 1 \implies g'(x) > 0
        \text{ for } x > 1 \text{ and } g'(x) < 0 \text{ for } x \in (0, 1)
    \end{align*}
    So, the derivative to the right of 1 is positive, and the derivative to the left of 1 is negative.
    \begin{center}   
        \begin{tikzpicture}
            \begin{axis}[
                axis x line=left,
                axis y line=left,
                axis line style={-{Stealth[scale=1.5]}},
                xtick={1},
                ytick=\empty,
                xmin=0, xmax=2, ymin=0, ymax=1.2,
                unit vector ratio=1 1 1,
                samples=100,
            ]
                \tikzset{graph/.style={very thick, smooth}}
                \addplot[blue, graph, domain=0:2] {(x - 1)^2};
                \addlegendentry{$g(x)$}
            \end{axis}
        \end{tikzpicture}
    \end{center}
    Therefore, $g(1) = 0$ is the minimum of $g$, thus:
    \begin{align*}
        &
        g(x) \ge 0 \quad \forall x \in (0, +\infty) \implies
        \frac{x^p}{p} + \frac{1}{q} \ge x
        \\&
        \text{Take } x = \frac{a}{b^{q - 1}}  \implies
        \frac{a^p}{pb^{p(q-1)}} + \frac{1}{q} \ge \frac{a}{b^{q-1}} \implies
        \frac{a^p}{b^{pq - p - q}} + \frac{b^q}{q} \ge ab
        \\&
        \frac{1}{p} + \frac{1}{q} = 1 \overset{\times pq}{\implies} p + q = pq \implies 
        pq - p - q = 0
        \\&
        \implies \frac{a^p}{p b^0} + \frac{b^q}{q} \ge ab \implies
        \frac{a^p}{p} + \frac{b^q}{q} \ge ab 
    \end{align*}
\end{proof}
\begin{remark}
    This function $g$ looks a bit artificial.
    It's not the only choice of such function, but it's not the key idea of
    the proof.
    The main idea is to create an inequality using derivatives (as we did).
\end{remark}
\begin{theorem}
    $E$ --- measurable, $1 \le p < \infty$, $q = \overline{p}$.
    If $f \in L^p(E)$ and $g \in L^q(E)$, then:
    \begin{enumerate}
        \item {
            $f \cdot g \in L^1(E)$ and
            \[
                \int_E \abs{fg} \le \norm{f}_p \cdot \norm{g}_q \qquad
                \text{(Hölder's inequality)}
            \]
        }
        \item {
            If $f$ is non-zero on a subset of positive measure (i.e.
            $[f] \ne [0]$), then
            \begin{align*}
                &
                f^*(x) \coloneqq \norm{f}_p^{1 - p} \cdot 
                \operatorname{sgn}(f(x)) \cdot \abs{f(x)}^{p-1} \in L^q(E),\ \norm{f^*}_q = 1
                \\&
                \text{and } \int_E f \cdot f^* = \norm{f}_p
            \end{align*}
        }
    \end{enumerate}
\end{theorem}
\begin{remark}
    $f^*$ is, in a sense, a rescaling of $\abs{f(x)^{p-1}}$.
    We multiply it by the sign function to preserve the sign of $f$.
\end{remark}
\begin{proof}
    \mbox{}
    \begin{enumerate}[label={Case \arabic*.}]
        \item {
            $p = 1$, $q = \infty$. In this case Hölder's inequality is obvious
            if we replace $g$ with its essential supremum:
            \[
                \int_E \abs{fg} \le 
                \int_E \abs{f} \cdot \operatorname{ess\,sup}_{E} \abs{g} = 
                \int_E \abs{f} \cdot \norm{g}_q =
                \norm{f}_p \cdot \norm{g}_q
            \]
            Now for the second part:
            \begin{align*}
                &p = 1 \implies f^*(x) = \operatorname{sgn}(f(x)) \in L^\infty(E)
                \overset{f^* \text{ is sgn}} \implies 
                \norm{f}_\infty = 1 \implies f \in L^\infty(E)
                \\&
                \text{and }
                \int_E f f^* = \int_E \abs{f} = \norm{f}_1
            \end{align*}
        }
        \item {
            $p, q \ne 1$. 
            Hölder's inequality is homogeneous (i.e. you can multiply or divide
            $f$ and $g$ by a constant and the inequality will still hold),
            therefore, without the loss of generality, we can assume that
            $\norm{f}_p = \norm{g}_q = 1$. Therefore, we need to prove that
            \[ \int_E \abs{fg} \le 1 \]
            Since $\abs{f}^p$ and $\abs{g}^q$ are integrable, it follows that
            $f$ and $g$ are infinite on subset of measure 0.
            Outside of that subset, by Young's inequality,
            \[
                \abs{fg} = \abs{f} \cdot \abs{g} \le
                \frac{\abs{f}^p}{p} + \frac{\abs{g}^q}{q}
                \text{ almost everywhere on } E
            \]
            Now let's put integrals on both sides:
            \[
                \int_E \abs{fg} \le
                \int_E \frac{\abs{f}^p}{p} + \int_E \frac{\abs{g}^q}{q}
                =
                \left[
                    \int_E \abs{f}^p = \norm{f}_p^p = 1^p = 1,\
                    \int_E \abs{g}^q = \norm{g}_q^q = 1^q = 1
                \right] = 
                \frac{1}{p} + \frac{1}{q} = 1
            \]

            Now let's do the second part.
            \begin{align*}
                &f f^* \overset{\operatorname{sgn}(f) \text{ cancels out}}{=}
                \norm{f}_p^{1 - p} \cdot \abs{f}^p 
                \text{ on } E \implies
                \int_E f f^* = \norm{f}_p^{1-p} \int_E \abs{f}^p =
                \norm{f}_p^{1-p} \cdot \norm{f}_p^p = \norm{f}_p
                \\&
                \int_E \abs{f^*}^q = 
                \norm{f}_p^{(1-p)q} \int_E \abs{f}^{(p-1)q} =
                \left[pq - p - q = 0\right] =
                \norm{f}_p^{-p} \cdot \int_E \abs{f}^p = 
                \norm{f}_p^{-p} \cdot \norm{f}_p^p = 1
                \\&
                \int_E \abs{f^*}^q = 1 \implies f^* \in L^q(E)
            \end{align*}
        }
    \end{enumerate}
\end{proof}
