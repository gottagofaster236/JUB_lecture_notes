\begin{proof}[Proof of Proposition~\ref{prop:propOneSeparability}]
    \mbox{}
    \begin{enumerate}
        \item {
            $p = \infty$. Let $g \in L^\infty(E)$. We want to approximate $g$
            with a sequence of simple functions that converge to $g$.
            $g \in L^\infty(E)$, thus, $g$ is bounded outside
            $E_0,\ m(E_0) = 0$.
            Now let's apply the \hyperref[lem:simpleApproxLemma]{simple approximation lemma}
            for $f$ restricted onto $E \setminus E_0$.
        }
        \item {
            $p < \infty$. Then we can't use the simple approximation lemma directly,
            as we don't have essential boundness. Let's use the
            \hyperref[the:simpleApproxTheorem]{simple approximation theorem instead}:
            there exists a sequence of simple functions $\{\varphi_n\}$, such that
            $\varphi_n \to g$ pointwise on $E$ and $\abs{\varphi_n(x)} \le \abs{g(x)}\ \forall n, x$.
            It follows that $\norm{\varphi_n}_p \le \norm{g}_p < \infty \implies
            \varphi_n \in L^p(E) \ \forall n$.

            Since $\abs{\varphi_n(x)} \le \abs{g(x)}$, we can also write that
            \[
                \abs{\varphi_n - g}^p \le (2 \abs{g})^p = 2^p \abs{g}^p
            \]
            Now we want to use the \hyperref[the:domConv]{dominated convergence theorem}.
            $g \in L^p(E)$, therefore, $\abs{g}^p$ is integrable, and multiplying it 
            by a constant ($2^p$) does not change that. So, we have dominated
            $\abs{\varphi_n - g}^p$ with $2^p \abs{g}^p$. Therefore, by dominated convergence theorem, we have:
            \[
                \lim_{n \to \infty} \int \abs{\varphi_n - g}^p = \int \abs{\lim_{n \to \infty} \varphi_n - g}^p = 0 
                \int \abs{g - g}^p = 0 \implies
                \lim_{n \to \infty} \int \abs{\varphi_n - g}^p = 0
            \]
            Therefore, $\varphi_n \to g$ in $L^p(E)$ by definition.
        }
    \end{enumerate}
\end{proof}

\begin{proof}[Proof of Proposition~\ref{prop:propTwoSeparability}]
    A simple function can be represented as 
    \[ \sum_{i=1}^n c_i \chi_{A_i} \]
    Let's approximate each of the characteristic functions, and then take
    the linear combination with $c_i$ as the coefficients.

    So, we want to approximate $\chi_A$ by a step function.
    $A$ is a measurable set of finite measure (since $[a, b]$ is bounded).
    
    Therefore, as we have proved previously,
    for any $\varepsilon > 0$ there exists a finite
    disjoint collection of open intervals $\{I_k\}$, such that if
    $U = \cup I_k$, then $m(U \triangle A) < \varepsilon$.
    Let's choose $\varepsilon = \varepsilon^p$, then $m(U \triangle A) < \varepsilon^p$ 
    instead. So:
    \[
        \norm{\chi_A - \chi_U}_p = \Bigl(\int_{A \triangle U} 1^p\Bigr) \le
        (\varepsilon^p)^{\frac{1}{p}} \le \varepsilon
    \]
\end{proof}

Now let's prove that $L^\infty[a,b]$ is not separable.
\begin{proof}
    Let $[a, b] = [0, 1]$.
    
    For every $y \in (0, 1)$, let's define $f_y = \chi_{[0, y]}$.
    If $y_1 \ne y_2$, then $\abs{f_{y_1}(x) - f_{y_2}(x)} = 1$ for every $x$
    between $y_1$ and $y_2$. Therefore, 
    $\norm{f_{y_1} - f_{y_2}}_\infty = 1$. 
    If we now take balls in $L^\infty[a,b]$ of radius $\frac{1}{2}$
    centered in those functions, they will not intersect (as the pairwise distances between centers are 1).
    Therefore, $L^\infty[a,b]$ is not separable, because we'd have to have at least one function 
    in every one of those balls, but there's countably many of them.
\end{proof}

\subsection{Completeness of $L^p$}
\begin{definition}
    $X$ --- normed vector space.
    
    \textit{Convergence}: 
    $f_n \to f$ if $\forall \varepsilon > 0 :\ \exists N: \forall n \ge N:\ \norm{f_n - f} < \varepsilon$.

    A sequence $\{f_n\}$ is \textit{Cauchy} if 
    $
        \forall \varepsilon > 0: \
        \exists N:\ \forall m, n \ge N:\ \norm{f_n - f_m} < \varepsilon
    $
\end{definition}
\begin{proposition}
    $X$ --- normed space. Then:
    \begin{enumerate}
        \item {
            Every convergent sequence is Cauchy.
        }
        \item {
            If a Cauchy sequence has a convergent subsequence, then
            if converges.
        }
    \end{enumerate}
\end{proposition}
\begin{proof}
    We'll leave the proof as an exercise. The main idea is that
    once we can guess the limiting function $f$,
    we can usually prove the convergence by definition.
\end{proof}
\begin{definition}
    A sequence $\{f_n\}$ in $X$ is called \textit{rapidly Cauchy}, if there
    exists a convergent series of positive numbers
    $\sum_1^\infty \varepsilon_k$, such that
    \[ \norm{f_{k+1} - f_k} \le \varepsilon_k^2 \quad \forall k \]
\end{definition}
\begin{proposition}
    \begin{enumerate}
        \item {
            Every rapidly Cauchy sequence is Cauchy.
        }
        \item {
            Every Cauchy sequence has a rapidly Cauchy subsequence.
        }
    \end{enumerate}
\end{proposition}
\begin{proof}
    \begin{enumerate}
        \item {
            If $\sum_1^\infty \varepsilon_k$ converges, then
            $\sum_1^\infty \varepsilon_k^2$ converges as well
            (as once the epsilons get less than one, that the squaring will only
            speed the convergence up).

            Therefore,
            \[
                \norm{f_n - f_m} \le \sum_n^{m - 1} \varepsilon_k^2
            \]
            and the tails of a convergent functions converge to 0.
        }
        \item {
            Each next element should be chosen sufficiently far in the line, 
            to ensure the consecutive differences are small enough.
        }
    \end{enumerate}
\end{proof}

\begin{theorem}
    $E$ --- measurable, $1 \le p \le \infty$. Then every rapidly Cauchy sequence
    in $L^p(E)$ converges both with respect to $\norm{\, \cdot \,}_p$
    and pointwise almost everywhere on $E$ to a function in $L^p(E)$.
\end{theorem}
\begin{proof}
    The case $p = \infty$ will be in the homework. Assume that $p < \infty$,
    $\{f_n\}$ --- rapidly Cauchy.
    Since our functions belong to the $L^p$ space, every such function
    is finite except a subset of measure zero. Let's throw this subset
    of measure zero out for every one of our functions.
    Since there's a countable number of functions, the set that
    we will throw out will still have measure 0.
    
    $\{f_n\}$ is rapidly Cauchy. Let's take the definition of a rapidly Cauchy
    sequence and exponentiate both sides to power $p$:
    \[
        \norm{f_{k+1} - f_k}_p = \Bigl(\int_E \abs{f_{k+1} - f_k}^p\Bigr)^\frac{1}{p}
        \le \varepsilon_k^{2} \quad \forall k \implies
        \int_E \abs{f_{k+1} - f_k}^p \le \varepsilon_k^{2p} \quad \forall k
    \]
    By \hyperref[the:cheb]{Chebyshev}, 
    \[
        m\{ x \in E \mathrel{\big\vert} \abs{f_{k+1}(x) - f_k(x)} \ge \varepsilon_k \} = 
        m\{ x \in E \mathrel{\big\vert} \abs{f_{k+1}(x) - f_k(x)}^p \ge \varepsilon_k^p \} \le
        \frac{1}{\varepsilon_k^p} \int_E \abs{f_{k+1}(x) - f_k(x)}^p \le \varepsilon_k^p
    \]
    $p \ge 1,\ \sum_1^\infty \varepsilon_k < \infty \implies \sum_1^\infty \varepsilon_k^p < \infty$.
    Now we can use the \hyperref[lem:borCantelly]{Borel–Cantelli lemma}.
    Let's say $E_0 \subset E$ is the subset of measure zero where Borel-Cantelli doesn't hold.
    Then every $x \in E \setminus E_0$ belongs to a finite number of such sets. Let's
    put $K(x)$ as the maximum index of such set plus one. Therefore:
    \[
        \abs{f_{k+1}(x) - f_k(x)} < \varepsilon_k \quad \forall k \ge K(x)
    \]
    Therefore, for every $x \in E \setminus E_0$, $\{f_k(x)\}$ is a Cauchy sequence.
    As a result, $f_k$ converges to some function $f$ pointwise on $E \setminus E_0$.

    Now we have a candidate for convergence, but we have to prove that it converges in the $L_p$ space.
    \[
        \int_E \abs{f_{n+k} - f_n}^p \le \sum_{m=n}^\infty \varepsilon_m^{2p} \quad 
        \forall n, k
    \]
    Take $k \to \infty$:
    \begin{align*}
        &
        \int_E \abs{f - f_n}^p \overset{\hyperref[lem:fatou]{\text{Fatou's lemma}}}{\le} 
        \liminf_{k \to \infty} \int_E \abs{f_{n+k} - f_n}^p
        \\&
        \norm{f - f_n}_p \le \Bigl( \sum_{m=n}^\infty \varepsilon_m^{2p} \Bigr)^{\frac{1}{p}}
    \end{align*}
    This is the tail of a converging series, and thus it converges to 0.
    Therefore, $f_n \to f$ in the $L^p$ norm.

    Why is $f \in L^p$? That's true as $L^p$ is a vector space,
    $f_n \in L^p, f - f_n \in L^p$. Then their sum is also in $L^p$:
    $f_n + (f - f_n) = f \in L^p$.
\end{proof}

Does pointwise convergence cause convergence in $L^p$, or vice versa?
It turns out that neither is true.
\begin{example}[1]
    \[ f_n = n^{1/p} \cdot \chi_{(0,\, 1/n]},\ E = [0, 1] \]
    $f_n \to 0$ pointwise. But:
    \[
        \int_E \abs{f_n - 0}^p = \int_0^{1/n} n = 1
    \]
\end{example}
\begin{example}[2]
    Let's imagine a piano.
    When we're playing a piano, we're pressing the keys one by one from left to right.
    Let's split the interval of $[0, 1]$ into $n$ pieces, and ``move from left to right''
    by selecting functions equal to $\chi_{[k / n,\, (k+1) / n]}$.
    We can imagine that when we press a key (which is $[k / n,\, (k+1) / n]$)  it is ``lifted up'' by 1
    (as opposed to being pushed down as on a real piano).

    When we've pressed every key from 1 to $n$ from left to right, let's subdivide
    each key into two, so now we have $2n$ keys. Now let's press every key from 1
    to $2n$ from left to right, and subdivide again, and so on.

    Each time the width of our piano keys will be smaller and smaller and smaller,
    and, therefore, the integral will converge to 0.
    However, there's no pointwise convergence on any subset of the interval,
    because every point on $[0, 1]$ will be ``played'' infinitely many times.
\end{example}