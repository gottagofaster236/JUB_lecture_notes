\begin{lemma}
    $U \subset \mathbb{R}^n$ --- any open subset.
    Every derivation on $C^\infty(U)$ at $p$ is some directional
    derivative evaluated at $p$.
\end{lemma}
\begin{remark}
    If $V$ is a chart of $M$, then it can be mapped to a subset of $\mathbb{R}^n$.
    Therefore, $C^\infty(V)$ can be mapped to $C^\infty(\mathbb{R}^n)$,
    so all derivations on $V$ are some directional derivatives.

    But since $V \subset M$, we have $C^\infty(V) \supset C^\infty(M)$
    (as every function that is smooth on $M$ will be smooth on a subset of $M$).
    Since there are less smooth functions on $M$ than on $V$, 
    there are less restrictions for a derivation, therefore, $M$
    could potentially have more derivations than $V$. (It can be proven though
    that that's not the case).
\end{remark}
\begin{proof}
    Without the loss of generality, let's assume that $p = \vec{0}$.
    If $f \in C^\infty(U)$, then by Taylor series:
    \[
        f(x) = f(0) + 
        \Bigl(\sum_{i=1}^n \frac{\partial f}{\partial x^i}\Big|_{x=0} \cdot x^i \Bigr)
        + R(x), \qquad R(x) = \mathcal{O}(\norm{x}^2)
    \]
    $R(x)$ is a linear combination of products $g(x) h(x)$, such that $g$, $h$ are smooth and $g(0) = h(0) = 0$.
    We can prove it in the following way. Let's take the tail of the Taylor series, $R(x)$. It has an infinite number of terms.
    Let's take all the terms that contain $x^1$, factor $x^1$ out, and say $g(x) = x^1$ and $h(x)$ is what's left.
    Not all terms contain $x^1$. Out of the remaining terms, let's find all that contain $x^2$, factor $x^2$ out, and so on.
    In the end, $R(x)$ will be a linear combination of no more than $n$ such products.
    
    If $D$ is an arbitrary derivation on $C^\infty(U)$ at 0, then:
    \begin{enumerate}
        \item[Step 1.] {
            $D(R) = 0$.
        }
        \item[Proof.] {
            By linearity, it's enough to prove that $D(gh) = 0$, where
            $g(0) = h(0) = 0$.
            \[ D(gh) = g(0) D(h) + h(0) D(g) = 0 + 0 = 0 \]
        }
        \item[Step 2.] {
            If $g = \text{const}$, then $Dg = 0$.
        }
        \item[Proof.] {
            Without the loss of generality, by linearity $g \equiv 1$. Then
            \[ D(g) = D(g^2) = 2g(0) D(g) = 2D(g) \implies D(g) = 0 \]
        }
        \item[Step 3.] {
            By steps 1 and 2, since $D$ is linear, we have:
            \[ 
                D(f) = \sum_{i=1}^n \frac{\partial f}{\partial x^i} \Big|_{x=0}
                \cdot D(x^i)
            \]
            Take $\vec{v} = [D(x^1), D(x^2), \dots, D(x^n)]$. Then
            $D(f) = (D_{\vec{v}} f)(0)$.
        }
    \end{enumerate}
\end{proof}

\begin{definition}[Differential]
    Let $M$, $N$ be smooth manifolds, $F : M \to N$ --- smooth, $p \in M$.
    The \textit{differential} of $F$ at $p$ (or \textit{pushforward}) is a map
    \[ dF_p : T_p M \to T_{F(p)} N \] defined by
    \[ dF_p(v)(f) = v(f \circ F) \text{ for every } v \in T_p M,\ f \in C^\infty(N) \]
\end{definition}
\begin{remark}
    The tangent spaces contain all derivations. So $v$ here is a derivation at $p$.
\end{remark}
\begin{remark}
    The motivation for calling this a differential is the following.
    If we zoom into a manifold at a point $p$, the manifold will start to look flat.
    Therefore, we can approximate the manifold at a point with its tangent space.
    In usual calculus, $\mathbb{R}^n$ coincides with its tangent space. 
    This is an unfortunate coincidence, that creates a lot of confusion.
\end{remark}
\begin{lemma}
    The definition is correct, i.e.
    $d F_p(v) \in T_{F(p)} N$ for all $v \in T_p M$.
\end{lemma}
\begin{proof}
    \begin{enumerate}
        \item {
            $d F_p(v)$ is linear on $C^{\infty(N)}$.
            This follows directly from the linearity of $v$.
        }
        \item {
            The product rule holds:
            \begin{align*}
                &
                d F_p(v)(fg) = v\bigl((fg) \circ F\bigr) = 
                v\bigl((f \circ F)(g \circ F)\bigr)
                =\\&=
                (f \circ F)(p) \cdot v(g \circ F) + 
                (g \circ F)(p) \cdot v(f \circ F) = 
                f(F(p)) \cdot dF_p(v)(g) + g(F(p)) \cdot dF_p(v)(f)
            \end{align*}
        }
    \end{enumerate}
\end{proof}

\begin{proposition}
    $M$, $N$, $P$ --- smooth manifolds,
    $F : M \to N$, $G : N \to P$ --- smooth maps, $p \in M$. Then:
    \begin{enumerate}
        \item {
            $dF_p : T_p M \to T_{F(p)} N$ is linear.
        }
        \item {
            $d(G \circ F)_p : T_p M \to T_{G \circ F(p)} P$ satisfies
            $d(G \circ F)_p = dG_{F(p)} \circ dF_p$. This is the chain rule.
        }
        \item {
            $d(\operatorname{id}_M) = \operatorname{id}_{T_p M}$.
        }
        \item {
            If $F$ is a diffeomorphism, then $dF_p$ is an isomorphism
            between $T_p M$ and $T_{F(p)} N$\\ and
            $(dF_p)^{-1} = d(F^{-1})_{F(p)}$.
        }
    \end{enumerate}
\end{proposition}

\subsection{Computations in charts}
Let $(U, \varphi)$ --- a chart at $p$, $\varphi : U \to \mathbb{R}^n;\ 
d\varphi_p : T_p M \to T_{\varphi(p)} \mathbb{R}^n$.

Here is the basis of $T_{\varphi(p)} \mathbb{R}^n$:
\[\Bigl(
    \frac{\partial}{\partial x^1}\Big|_{\varphi(p)}, 
    \frac{\partial}{\partial x^2}\Big|_{\varphi(p)}, \dots
    \frac{\partial}{\partial x^n}\Big|_{\varphi(p)}
 \Bigr) \]
We can pull this basis back to $T_p M$ using the differential of $\varphi$.
Let's denote it as follows:
\[
    \Bigl(
        \frac{\partial}{\partial x^1}\Big|_p,
        \frac{\partial}{\partial x^2}\Big|_p, \dots
        \frac{\partial}{\partial x^n}\Big|_p
    \Bigr) 
    \text{ where }
    \frac{\partial}{\partial x^i}\Big|_p f =
    \frac{\partial}{\partial x^i}\Big|_{\varphi(p)} (f \circ \varphi^{-1}) =
    \frac{\partial \hat{f}}{\partial x^i}(\hat{p}) 
\]
$\hat{f}$ and $\hat{p}$ are the representations of $f$ and $p$ in chart $(U, \varphi)$.
This basis will be different for a different chart.

Let $F : U \to V$, where $U \subset \mathbb{R}^n$, $V \subset \mathbb{R}^m$.
Let's check that the definition of the differential that we gave
will be the Jacobian in this case.

Let $(x^1, \dots, x^n)$ be the standard basis of $\mathbb{R}^n$, 
$(y^1, \dots, y^n)$ --- the standard basis of $\mathbb{R}^m$,
$F = (F^1, \dots, F^m)$.
It is enough to check how the differential acts on the basis vectors.
\begin{align*}
    &
    dF_p\Bigl(\frac{\partial}{\partial x^i} \Big|_p\Bigr) f 
    \overset{\text{definition of differential}}{=}
    \left[\frac{\partial}{\partial x^i} \Big|_p \text{is a basis vector in }
     T_p U = T_p \mathbb{R}^n\right] 
    =
    \frac{\partial}{\partial x^i} \Big|_p (f \circ F)
    \overset{\text{chain rule}}{=}\\&=
    \sum_{j=1}^m \frac{\partial f}{\partial y^j}(F(p))
    \cdot \frac{\partial F^j}{\partial x^i}(p) = 
    \Bigl(\sum_{j=1}^m \frac{\partial F^j}{\partial x^i} \cdot
    \frac{\partial}{\partial y^j}\Big|_{F(p)}\Bigr) f
    \text{ --- basis vector in } T_{F(p)} V
\end{align*}
This corresponds to the $i$-th column vector in the Jacobi matrix:
\[
    J_F = \begin{bmatrix}
        \frac{\partial F^1}{\partial x^1}(p) & \dots & \frac{\partial F^1}{\partial x^n}(p)\\
        \vdots & \ddots & \vdots \\
        \frac{\partial F^m}{\partial x^1}(p) & \dots & \frac{\partial F^m}{\partial x^n}(p)\\
    \end{bmatrix}
\]
$dF_p$ acts on vectors as left multiplication by $J_F$.
\subsubsection*{General case}
$F : M \to N$. $(U, \varphi)$ --- chart on $M$, $(V, \psi)$ --- chart on $N$.
$\hat{F} = \psi \circ F \circ \varphi^{-1};\ \hat{p} = \varphi(p)$.
\[
    dF_p\Bigl(\frac{\partial}{\partial x^i}\Big|_p\Bigr) = 
    \sum_{j=1}^m \frac{\partial \hat{F}^j}{\partial x^i}(\hat{p}) 
    \cdot \frac{\partial}{\partial y^j}\Big|_{F(p)}
\]

\subsection{Informal statements}
\begin{definition}[Tangent bundle]
    \[ TM \coloneqq \sqcup_{p \in M} T_p M  = \{
        (p, v): p \in M, v \in T_p M
    \}\]
    In other words, we think of different tangent spaces independently,
    as a collection.
\end{definition}
$TM$ --- smooth manifold with charts
\[ 
    \biggl(\Bigl(U, \sqcup_{p \in U} T_p M\Bigr),\ (\varphi, d_p \psi)\biggr)
\]
Here $U$ is a chart in $M$.
To verify that this is a smooth manifold, we have to make sure that the
transition maps are smooth.
If we have another chart $V \in M$ with transition function
$(\psi, d_p \psi)$ then the transition map will be
\[ 
    (\psi \circ \phi^{-1}, d_p \psi \circ (d_p \phi)^{-1}) = 
    (\psi \circ \phi^{-1}, d_{\phi(o)}(\psi \cdot \phi^{-1}))
\]