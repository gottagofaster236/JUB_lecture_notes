%\subsection{Idk}

\begin{definition}[Numerical Methods]
    Numerical Methods are algorithmic approaches to numerically solve mathematical problems.
    We use it often it is hard/difficult/impossible to solve analytically.
\end{definition}

\subsection{Taylor series}
Given a function $f : \mathbb{R} \to \mathbb{R}$
(that is hard to evaluate for some $x \in \mathbb{R}$),
but $f$ and $f^{(n)}$ are known for a value $c$, which is close to $x$.
Can we use this information to approximate $f(x)$?

We know values for $\cos^{(n)}(0)$.
\[
    \begin{cases}
        f(0) = \cos(0) = 1\\
        f'(0) = -\sin(0) = 0\\
        f''(0) = -\cos(0) = -1\\
    \end{cases} \text{ for } c = 0
\]
Can we get $\cos(0.1)$ from this?

\begin{definition}[Taylor series]
    Let $f : \mathbb{R} \to \mathbb{R}$, differentiable 
    infinitely many times at $c \in \mathbb{R}$.
    So we have $f^{(k)}(c),\ k=1,2,\dots $. Then the Taylor series of $f$ at $c$ is:
    \[
        f(x) \approx f(c) + \frac{f(c)}{1!}(x-c)^1 + \frac{f''(c)}{2!}(x-c)^2 + \dots =
        \sum_{k=0}^{\infty} \frac{f^{(k)}}{k!} (x - c)^k
    \]
\end{definition}

\begin{remark}
    Taylor series is a power series.
\end{remark}

\begin{remark}
    For $c = 0$ also known as Maclaurin series
\end{remark}

\begin{remark}
    A power series has an interval/radius of convergence.
    You can only evaluate the series if $x \in \text{interval of convergence}$.
\end{remark}

\begin{example}[1]
    What is the Taylor series for $f(x) = e^x$ at $c = 0$?
    We have $f^{(k)}(x) = e^x$, so $f^{(k)}(0) = 1$.
    Thus: \[
        \sum_{k = 0}^{\infty} \frac{1}{k!} x^k
    \]
    and the radius of convergence is $\infty$.

    I.e. for any $x \in \mathbb{R}$:
    \[e^x = \sum_{k = 0}^{\infty} \frac{x^k}{k!}\]

    For an algorithm we need a finite amount of terms. For example,
    \[
        e^x \approx \frac{1}{0!} x^0 + \frac{1}{1!} x^1 + \frac{1}{2!} x^2 =
        1 + x + \frac{x^2}{2}
    \]
    This is a polynomial!
\end{example}

\begin{example}[2]
    Let's calculate Taylor series of a polynomial.
    \begin{align*}
        &
        f(x) = 4x^2 + 5x + 7,\ c = 2
        \\&
        f(2) = 33,\ f'(2) = 8x + 5\Big|_{x = 2} = 21,\ f''(2) = 8
    \end{align*}
    Taylor series:
    \[
        33 + 21(x - 2) + \frac{8}{2} (x - 2)^2 = 4x^2 + 5x + 7 = f(x)
    \]
    Taylor series of a polynomial is itself.
\end{example}

\begin{theorem}[Taylor theorem]
    Let $f \in \mathrm{C}^{n + 1}([a, b])$ (i.e. $f$ is $(n+1)$-times
    continuously differentiable).
    Then for any $x \in [a, b]$ we have that 
    \[
        f(x) = \sum_{k=0}^n \frac{f^{(k)}(c)}{k!}(x - c)^k +
        \frac{f^{(n+1)}(\xi_x)}{(n+1)!} (x - c)^{n+1}
    \]
    where $\xi_x$ is a point that depends on $x$ and which is between

    The first sum is called \textit{truncated Taylor series}, the error is called \textit{the remainder}.
\end{theorem}