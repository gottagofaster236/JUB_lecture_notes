\textbf{Summary of Taylor series:}
\begin{itemize}
    \item {
        Problem: Evaluate $f(x)$ with a given error bound.
    }
    \item {
        Required: $f \in \mathrm{C}^{n+1}$, values of derivatives $f^{(k)}(a)$.
    }
    \item {
        Check interval of convergence: does Taylor series expansion work?
    }
    \item {
        Estimate the maximum error for $n$ terms of the Taylor polynomial.
    }
    \item {
        Choose $n$, such that the error bound is low enough.
    }
    \item {
        Evaluate the Taylor polynomial.
    }
\end{itemize}

\newpage
\subsection{Number representation}

There are different error types:
\begin{enumerate}
    \item {
        An error in data (partly due to roundoff).
    }
    \item {
        Roundoff errors (during computation). For example,
        multiplication increases the amount of needed significant digits, and
        we can't store them all on a computer.
    }
    \item {
        Truncation error, that is inherent to numeric methods. For example,
        if we take a finite number of terms in our Taylor series.
    }
\end{enumerate}

\begin{definition}
    Let $\tilde{a}$ be an approximation of $a$. Then
    $\abs{\tilde{a} - a}$ is the \textit{absolute error}, and 
    $\abs{\frac{\tilde{a} - a}{a}}$ is the \textit{relative error}.
    The \textit{error bound} is the magnitude of admissable error.
\end{definition}
\begin{example}
    0.00123 with error $0.000004 = 0.4 \cdot 10^{-5}$.
    The error is below $\frac{1}{2} \cdot 10^{-t}$ with $t = 5$, so there's 5
    correct digits and 3 significant digits (the number of non-leading zeros).
\end{example}
\begin{example}
    0.00123 with error $0.000006 = 0.06 \cdot 10^{-4} > \frac{1}{2} \cdot 10^{-5}$.
    Only has 2 significant digits, because we have to round the error up.
\end{example}

\begin{theorem}
    In addition/subtraction the bounds for absolute errors are added,
    in multiplication/division the relative errors are added.
\end{theorem}
\begin{example}
    Solve $x^2 - 56x + 1 = 0$:
    \[
        x = 28 - \sqrt{783} \approx 28 - 27.982 \text{ (5 significant digits)} =
        0.018 \pm \frac{1}{2} \cdot 10^{-3}
    \]
    We end up with 2 significant digits in the answer, despite that we used to 
    have 5. That's why computers use floating point numbers, as
    leading zeros are bad.
\end{example}

\begin{definition}[Error propagation]
    If $y(x)$ is smooth, than the derivative $\abs{y(x)'}$
    can be interpreted as the sensitivity of $y(x)$ to errors in $x$.
    We can generalize this to functions of multiple variables:
    \[
        \abs{\Delta y} \le \sum \abs{\frac{\partial y}{\partial x_i}} \cdot \abs{\Delta x_i}
        \text{, where } \Delta y = \tilde{y} - y,\ \Delta x_i = \tilde{x_i} - x_i
    \]
    This is an empirical inequality that is only valid for small $\Delta x_i$.
    It is used a lot in physics.
\end{definition}

\begin{definition}[Base representation]
    Every number $x \in \mathbb{N}$ can be written in the following form 
    as a unique expansion with respect to the base $b$, where
    $b \in \mathbb{N} \setminus \{1\}$:
    \begin{align*}
        &
        x = a_0 b^0 + a_1 b^1 + a_2 b^2 + \dots + a_n b_n = 
        \sum_{i=0}^n a_i b^i 
        \\&
        a \in \mathbb{N}_0,\ a_i < b,\ a_i \in \{0, \dots, b - 1\}
    \end{align*}
        
    Here $b$ is called the \textit{base}, $a_i$ are called the \textit{digits}.
    Humans usually use base 10. But, for example, computers
    can use base 2.

    For a real number $x \in \mathbb{R}$ we can write: 
    \[
        x = \sum_{i=0}^n a_i b^i + \sum_{i=1}^\infty a_{-i} b^{-i}
    \]
\end{definition}
\begin{example}
    $b=2:\ 1011 = 1 \cdot 2^0 + 1 \cdot 2^1 + 0 \cdot 2^2 + 1 \cdot 2^3 =
    (11)_{10}$
\end{example}
There are different algorithms that convert number systems.

\subsubsection{Euclid's algorithm}
Euclid's algorithm converts $(x)_{10}$ to $(y)_b$.
\begin{enumerate}%[label={\arabic*)}]
    \item {
        Input $(x)_{10}$.
    }
    \item {
        Determine the smallest $n$, such that $x < b^{n+1}$.
    }
    \item {
        For $i=n$ to $0$ do:
        \begin{align*}
            &
            a_i := x\ \mathrm{div}\ b^{i} \text{ (integer division)}
            \\&
            x := x\ \mathrm{mod}\ b^{i} \text{ (the remainder)}
        \end{align*}
    }
    \item {
        Output result $a_n a_{n-1} a_{n-2} \dots a_0 = (y)_b$.
    }
\end{enumerate}

\begin{example}
    \begin{enumerate}
        \item {
            $(x)_{10} = (13)_{10} \to (y)_2$
        }
        \item {
            $n = 3$ since $13 < 2^4$.
        }
        \item {
            \begin{align*}
                &
                i = 3:\ a_3 = 13 \ \mathrm{div}\ 2^3 = 1,\
                x = 13 \ \mathrm{mod}\ 2^3 = 5
                \\&
                i = 2:\ a_2 = 5 \ \mathrm{div}\ 2^2 = 1,\
                x = 5 \ \mathrm{mod}\ 2^2 = 1
                \\&
                i = 1:\ a_1 = 1 \ \mathrm{div}\ 2^1 = 0,\
                x = 1 \ \mathrm{mod}\ 2^1 = 1
                \\&
                i = 0:\ a_0 = 1 \ \mathrm{div}\ 2^0 = 1,\
                x = 1 \ \mathrm{mod}\ 2^0 = 0
            \end{align*}
        }
        \item {
             Output: $(1101)_2 = (13)_{10}$.
        }
    \end{enumerate}
\end{example}
Two problems of the Euclid's algorithm:
\begin{enumerate}
    \item {
        Step 2 is inefficient
    }
    \item {
        Division by large numbers can be problematic.
    }
\end{enumerate}

\subsubsection{Horner's scheme}
Horner's scheme is a more efficient algorithm. The idea is to represent
the number as follows:
\[
    (a_n a_{n - 1} \dots a_{0})_b = 
    a_0 + b\bigl(a_1 + b(a_2 + b(a_3 + \dots + b (a_n))) \dots \bigr)
\]
The algorithm is the following:
\begin{enumerate}
    \item {
        Input $(x)_{10}$.
    }
    \item {
        $i := 0$.
    }
    \item {
        While $x > 0$ do:
        \begin{align*}
            &
            a_i := x \ \mathrm{mod}\ b
            \\&
            x := x \ \mathrm{div}\ b
            \\&
            i := i + 1
        \end{align*}
    }
    \item {
        Output result $a_n a_{n-1} a_{n-2} \dots a_0 = (y)_b$.
    }
\end{enumerate}
\begin{remark}
    The algorithm is very similar to the Euclid's algorithm ---
    the difference is that we execute it in reverse. 
    We no longer have divisions by large numbers, and thus
    it runs faster.
\end{remark}

\textbf{General remarks:}
\begin{enumerate}
    \item {
        A number with simple representation in one base 
        may be complicated to represent in another base.
        For example,
        $(0.1)_{10} = (0.0001100110011\dots)_2$.
    }
    \item {
        Base 2 is called \textit{binary},
        base 8 is \textit{octal},
        base 16 is \textit{hexadecimal}.
    }
    \item {
        To convert from a base $b$
        to base $10$ we can just perform the following computation:
        \[
            (42)_8 = 4 \cdot 8^1 + 2 \cdot 8^0 = (34)_{10}
        \]
    }
\end{enumerate}