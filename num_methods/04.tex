\subsection{Floating point representation}
\begin{definition}
    Normalized floating point representation
    with respect to a base $b$ stores any number $x$ as follows:
    \[
        x = 0.a_1 a_2 \dots a_k \cdot b^n \text{ where } a_i \in \{0, 1, \dots, b - 1\}
    \]
    $a_i$ are called \textit{digits}, $k$ is called \textit{precision},
    $n$ is called \textit{exponent}, $a_1 \dots a_k$ is called \textit{mantissa},
    $a_1 \ne 0$ is called \textit{normalization}, which makes the representation
    unique.
\end{definition}
\begin{remark}
    Leading zeros are essentially a waste of space. That's why floating points
    are useful when you add/subtract/divide numbers --- as you're able to move
    the decimal point. 
\end{remark}
\begin{example}[1]
    Base 10: $32.213 = 0.32213 \cdot 10^2$.

    Base 2: $x = \pm 0.b_1 b_2 \dots b_k \cdot 2^n$
    We need another bit to define the sign of the number.
\end{example}
\begin{example}[2]
    For \textit{single precision} floating-point numbers:
    4 bytes = 32 bits.
    \begin{itemize}
        \item {
            1 bit sign mantissa.
        }
        \item {
            1 bit sign exponent.
        }
        \item {
            7 bits for exponent (integer).
        }
        \item {
            23 bits for mantissa (24 effectively, since the first digit is always 1).
        }
    \end{itemize}

    7 bits allow for the largest exponent of 127, i.e. 
    $2^{127} \approx 10^{38}$. Anything above that is infinity. So, we have the range of
    $10^{-38} < \abs{x} < 10^{38}$.

    Since $2^{-24} \approx 10^{-7}$, we can represent 7 significant digits.

    We can have a better representation, e.g. with \textit{double precision} (8 bytes).
\end{example}

Issues with floating-point numbers:
\begin{itemize}
    \item {
        Adding numbers is not commutative, i.e.
        $x + y$ does not necessarily equal to $y + x$.
    }
    \item {
        It's not associative, i.e.
        $(x + y) + z$ does not necessarily equal to $x + (y + z)$.
        You usually try to add small numbers together first, in hopes that they 
        will become big enough to become significant.
    }
\end{itemize}
\begin{example}[1]
    Take $x = y = 0.00000033$, $z = 0.00000034$, $w = 1.00000000$.
    We compute $x + y + z + w$, and in that order we get $1.000001$.
    Reversed order gets $1.000000$ with base 10. Why is that?
    That's because we only have $7$ significant digits.
\end{example}
\begin{example}[2]
    \begin{align*}
        \sum_{1}^{10^7} 1 + 10^7 = 1 + \dots + 1 + 10^7
    \end{align*}
    The order of summation will make a difference. You will either get 
    $10^7$ or $2 \cdot 10^7$, because $1 + 1 = 2$, but $1 + 10^7 = 10^7$
    due to roundoff.
\end{example}
\begin{consequence}
    Avoid adding numbers of different order of magnitude.
    Add numbers in increasing order of their size.
\end{consequence}
\begin{example}[3]
    Compute $x - \sin{x}$ for $x$ close to $0$, e.g. $x = 1 / 15$.
    Assume $k = 10$ precision.
    \begin{align*}
        &
        x = 0.6666666667 \cdot 10^{-1}
        \\&
        \sin{x} = 0.6661729492 \cdot 10^{-1}
        \\&
        x - \sin{x} = 0.0004937175 \cdot 10^{-1} = 
        0.4937175\underline{000} \cdot 10^{-4}
    \end{align*}
    Notice the three zeros at the end --- that is a sign of 
    a precision loss (unless the number actually ends with zeros).
    \begin{consequence}
        Avoid subtracting numbers of similar size, because it leads to
        a loss of precision.
    \end{consequence}
    A potential solution in this case would be to use Taylor series expanison:
    \begin{align*}
        &
        \sin{x} = x - \frac{x^3}{3!} + \frac{x^5}{5!} - \dots
        \\&
        x - \sin{x} = +\frac{x^3}{3!} - \frac{x^5}{5!} + \dots
    \end{align*}
    Using 3 terms we get: $0.4937174328 \cdot 10^{-4}$.
    The error of the Taylor series with 3 terms will be 
    $\le 10^{-13}$ (which is needed for ``full'' precision for $0.\ldots \cdot 10^{-4}$).
\end{example}
\begin{theorem}
    Let $x, y$ be two normalized floating point numbers
    with $x > y > 0$ and base $b=2$. If there exist $p, q \in \mathbb{N}_0$ such that
    \[ 2^{-p} \le 1 - \frac{y}{x} \le 2^{-q} \]
    Then at most $p$ and at least $q$ significant bits (digits base 2) are lost during subtraction.
\end{theorem}
