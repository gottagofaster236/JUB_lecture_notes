\pagebreak
\section{Linear Systems of Equations}
\begin{definition}
    A linear system of equations is given by
    \[ 
        Ax = b,\ A \in \mathbb{R}^{m \times n},\ x \in \mathbb{R}^n,
        b \in \mathbb{R}^m
    \]
    I.e. the matrix $A$ has $m$ rows and $n$ columns, $x$ is a vector with
    $n$ unknowns, $b$ has $m$ entries, thus the system has
    $m$ equations.
\end{definition}
\begin{remark}
    The system of equations is called \textit{linear}, because 
    the degree of all $x_i$ is equal to one.
\end{remark}
\begin{remark}
    If $n = m$, the system is called \textit{square}. (As the matrix is square).
\end{remark}
\begin{remark}
    We can also write the system as a sum:
    \[
        \sum_{j=1}^n a_{ij} x_j = b_i,\ i = 1, \dots, m
    \]
\end{remark}

\begin{example}
    \[
        \begin{cases}
            a_{11} x_1 + a_{12} x_2 = b_1\\
            a_{21} x_1 + a_{22} x_2 = b_2
        \end{cases}\
        n = 2,\ m = 2
    \]
\end{example}

Linear systems of equations arise in a lot of problems:
\begin{itemize}
    \item {
        Geometrical problems (coordinate transforms, 3D matrices).
    }
    \item {
        Electrical circuits, Kirchhoff's laws/Ohm's laws.
    }
    \item {
        Solving differential equations.
    }
    \item {
        GPS.
    }
\end{itemize}

\subsection{Gaussian Elimination (GE)}
Assume that $m = n$ (square system).
The idea of Gaussian Elimination: do row operations to produce an 
upper triangular matrix (echelon form). Then do backward substitution
to solve the system.

Allowed row operations:
\begin{enumerate}
    \item {
        Swap rows.
    }
    \item {
        Scale rows, i.e. multiply a row by a scaler.
    }
    \item {
        Add multiples of one row to another.
    }
\end{enumerate}

\begin{example}
    \[
        A = \begin{bmatrix}
            6 & -2 & 2 & 4\\
            12 & -8 & 6 & 10\\
            3 & -13 & 9 & 3\\
            -6 & 4 & 1 & -18
        \end{bmatrix},\quad
        b = \begin{bmatrix}
            16 \\ 26 \\ -19 \\ -34
        \end{bmatrix}
    \]

    \begin{enumerate}[font={\bfseries}, label={Step \arabic*.}]
        \item {
            Do GE in a systematic way:
            \[
                \text{Augmented matrix} =
                \left[\begin{array}{cccc|c}
                    \textcolor{red}{6} & -2 & 2 & 4 & 16\\
                    12 & -8 & 6 & 10 & 26\\
                    3 & -13 & 9 & 3 & -19\\
                    -6 & 4 & 1 & -18 & -34
                \end{array}\right]
            \]
            The \textcolor{red}{6} here is the pivot element, and the first row
            is the pivot row.
            \begin{align*}
                &\left[\begin{array}{cccc|c}
                    \textcolor{red}{6} & -2 & 2 & 4 & 16\\
                    12 & -8 & 6 & 10 & 26\\
                    3 & -13 & 9 & 3 & -19\\
                    -6 & 4 & 1 & -18 & -34
                \end{array}\right]
                \textcolor{red}{\begin{array}{l}
                    \leftarrow\ \text{pivot row}\\
                    \leftarrow\ (-2) \cdot R_1 + R_2 \\
                    \leftarrow\ (-1/2) \cdot R_1 + R_3 \\
                    \leftarrow\ 1 \cdot R_1 + R_4
                \end{array}}
                \\
                \hookrightarrow
                &\left[\begin{array}{cccc|c}
                    6 & -2 & 2 & 4 & 16\\
                    0 & \textcolor{red}{-4} & 2 & 2 & -6\\
                    0 & \textcolor{blue}{-12} & 8 & 1 & -27\\
                    0 & \textcolor{green}{2} & 3 & -14 & -18
                \end{array}\right]
                \textcolor{red}{\begin{array}{l}
                    \\
                    \leftarrow\ \text{pivot row}\\
                    \leftarrow\ (-3) \cdot R_2 + R_3 \\
                    \leftarrow\ (1/2) \cdot R_2 + R_4
                \end{array}}
            \end{align*}
            Always consider the factor, e.g.
            \begin{align*}
                -3 &= -\Bigl(\frac{\textcolor{blue}{-12}}{\textcolor{red}{-4}}\Bigr)
                \\
                \frac{1}{2} &= -\Bigl(\frac{\textcolor{green}{2}}{\textcolor{red}{-4}}\Bigr)
            \end{align*}
            Eventually, we end up with a triangular form (using diagonal elements as pivots).
            \[
                \left[\begin{array}{cccc|c}
                    6 & -2 & 2 & 4 & 16\\
                    0 & -4 & 2 & 2 & -6\\
                    0 & 0 & 2 & -5 & -9\\
                    0 & 0 & 0 & -3 & -3
                \end{array}\right]
            \]
        }
        \item {
            Backward substitution:
            \begin{itemize}
                \item {
                    Last row: $-3 x_4 = -3 \Longleftrightarrow x_4 = 1$.
                }
                \item {
                    Second last row:
                    \begin{align*}
                        &2 x_3 - 5x_4 = -9\\
                        &2x_3 - 5 = -9 \Longleftrightarrow x_3 = -2
                    \end{align*}
                }
                \item {
                    \dots finally:
                    $x_1 = 3,\ x_2 = 1,\ x_3 = -2,\ x_4 = 1$.
                }
            \end{itemize}
        }
    \end{enumerate}
\end{example}

The algorithm again:
\begin{enumerate}
    \item {
        Input $A \in \mathbb{R}^{n \times n},\ b \in \mathbb{R}^n$.

        \textbf{Forward substitution:}
    }
    \item {
        For $k = 1, \dots, n - 1$ (for all pivot rows, except the last one):
    }
    \item {
        \hspace*{0.5cm}  For $i = k + 1, \dots, n$ (for all rows below the pivot row):
    }
    \item {
        \hspace*{1cm}  For $j = k, \dots, n$ (for all columns from the pivot one):\\
        \hspace*{1.5cm} $
            a_{ij} \coloneqq a_{ij} - \frac{a_{ik}}{a_{kk}} a_{kj}
        $\\
        \hspace*{1cm} End for.\\
        \hspace*{1cm} $
            b_i \coloneqq b_i - \frac{a_{ik}}{a_{kk}} b_k
        $
    }
    \item[$\overline{3}$.] {
        \hspace*{0.5cm} End for.
    }
    \item[$\overline{2}$.] {
        End for.

        \textbf{Backward substitution:}
    }
    \item {
        $
            x_n = \frac{b_n}{a_nn} \text{ (last unknown)}
        $
    }
    \item {
        For $i = n-1, \dots, 1$ (return back row by row)\\
        \hspace*{0.5cm} $\mathrm{rhs} \coloneqq b_i$
    }
    \item {
        \hspace*{0.5cm}  For $j = n, \dots, i + 1$ (for all columns up to the pivot element)\\
        \hspace*{1cm}  $\mathrm{rhs} \coloneqq \mathrm{rhs} - a_{ij} x_i$ (all $x_j$ are already known)
    }
    \item[$\overline{7}$.] {
        \hspace*{0.5cm} End for.
        \hspace*{0.5cm} $x_i \coloneqq \frac{\mathrm{rhs}}{a_{ii}}$
    }
    \item[$\overline{6}$.] {
        End for.
    }
\end{enumerate}
