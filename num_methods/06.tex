GE can be used whenever the pivots don't vanish.
\begin{example}
    \[
    \begin{cases}
        x_1 + x_2 + x_3 = 1\\
        x_1 + x_2 + 2x_3 = 2\\
        x_1 + 2x_2 + 2x_3 = 1
    \end{cases} \implies
    \begin{cases}
        x_1 = 1\\ x_2 = -1 \\ x_3 = 1
    \end{cases}
\]

But addition of rows will give us:
\[
    \begin{cases}
        x_3 = 1 \text{ -- here we have a missing pivot}\\
        x_2 + x_3 = 0
    \end{cases}
    \qquad
    \left(\begin{array}{ccc|c}
        0 & 0 & 1 & 1\\
        0 & 1 & 1 & 0\\
        1 & 2 & 2 & 1
    \end{array}
    \right)
\]
\end{example}

We already get into trouble with
very small pivot elements.
\begin{example}
    Let $\varepsilon > 0$ and consider
    \[
        \begin{cases}
            \varepsilon x_1 + x_2 = 1\\
            x_1 + x_2 = 2
        \end{cases}
        \Longleftrightarrow
        \begin{pmatrix}
            2 & 1\\
            1 & 1
        \end{pmatrix}
        \vec{x} = \begin{pmatrix}
            1 \\ 2
        \end{pmatrix}
    \]
    For $\varepsilon \ll 1$, the actual solution is $x_1 \approx x_2 \approx 1$.
    However, GE yields
    \[
        x_2 = \frac{2 - \frac{1}{\varepsilon}}{1 - \frac{1}{\varepsilon}}
        \overset{\varepsilon \ll 1}{\approx}
        \frac{-\frac{1}{\varepsilon}}{-\frac{1}{\varepsilon}} = 1
    \] 
    WIth finite precision we will get through backward substitution:
    $x_2 = 1$ and $x_1 = \frac{1 - x_2}{\varepsilon} = 0$ which is wrong.
    The pivot is too small. But change order of equations.
    \[
        \begin{cases}
            x_1 + x_2 = 2\\
            \varepsilon x_1 + x_2 = 1
        \end{cases}
        \overset{\text{GE}}{\implies}
        \begin{cases}
            x_2 = \frac{1 - 2\varepsilon}{1 - \varepsilon}\\
            x_1 = 2 - x_2
        \end{cases}
    \]
    Now the answer is correct. The reason why the first one was incorrect is
    error amplification of $x_2$ by multiplication.
    $\frac{1}{\varepsilon}$ leads in the first case to a wrong result.
\end{example}

\pagebreak
\subsection{Scaled partial pivoting}
\begin{definition}
    \textit{Pivoting} means that the pivot element is chosen 
    appropriately, and not just row by row.
\end{definition}
\begin{definition}
    \textit{Partial pivoting} means we will reorder rows
    (not columns, otherwise it would be full pivoting).
\end{definition}
\begin{definition}
    \textit{Scaled} means we look for best \textit{relative} pivot,
    i.e. best ratio between pivot element and maximal entry of row
    (all in absolute values).
\end{definition}
\begin{remark}
    This will lead to minimal error propagation.
\end{remark}

The algorithm:
\begin{enumerate}
    \item {
        Input $A \in \mathbb{R}^{n \times n},\ b \in \mathbb{R}^m$.
    }
    \item {
        Find maximal absolute values of entries in rows
        $s \in \mathbb{R}^n$, such that $s_i = \max_{j=1}^n \abs{a_{ij}}$.

        Forward elimination:
    }
    \item {
        For $k = 1, \dots, n - 1$  (for all pivot rows).
    }
    \item {
        \hspace*{0.5cm} For $i = k, \dots, n$ (for all rows below pivot row)\\
        \hspace*{1cm} compute $\abs{\frac{a_{ik}}{s_i}}$.
    }
    \item[$\overline{4}$.] {
        \hspace*{0.5cm} End for.
    }
    \item {
        \hspace*{0.5cm} Find row with the largest relative pivot element, 
        name it row $j$.
    }
    \item {
        \hspace*{0.5cm} Swap $k$ with $j$.
    }
    \item {
        \hspace*{0.5cm} Swap entries $k$ and $j$ in vector $s$.
    }
    \item {
        \hspace*{0.5cm} Do skip of forward elimination in row $k$.
    }
    \item[$\overline{3}$.] {
        End for.
    }
\end{enumerate}
Backward substitution is done as before, but with updated order.
\begin{example}
    \[
        \left[
        \begin{array}{cccc|c}
            3 & -13 & 9 & 3 & -19\\
            -6 & 4 & 1 & -18 & -32\\
            6 & -2 & 2 & 4 & 16\\
            -12 & -8 & 6 & 10 & 26
        \end{array}
        \right]
    \]
    Initial $s = (13, 18, 6, 12)$.
    Iterations:
    \begin{enumerate}
        \item {
            \begin{itemize}
                \item {
                    Relative pivots:
                    \[
                        \Bigl(
                            \frac{3}{13}, \frac{6}{18},
                            \frac{6}{6}, \frac{12}{12}
                        \Bigr) =
                        \biggl(\abs{\frac{a_{ik}}{s_i}}\biggr)
                    \]
                }
                \item {
                    Rows 3 and 4 have pivot 1 greater than all others.
                    Select for swapping rows 1 and 3.
                    \[
                        \left[\begin{array}{cccc|c}
                            6 & -2 & 2 & 4 & 16\\
                            -6 & 4 & 1 & -18 & -32\\
                            3 & -13 & 9 & 3 & -19\\
                            12 & -8 & 6 & 10 & 26
                        \end{array}\right]
                    \]
                }
                \item {
                    Swap entries $3 \leftrightarrow 1$ in $s$:
                    $(6, 18, 13, 12)$.
                }
                \item {
                    Forward elimination step (like in GE):
                    \[
                        \left[\begin{array}{cccc|c}
                            6 & -2 & 2 & 4 & 16\\
                            0 & 2 & 3 & -14 & -18\\
                            0 & -12 & 8 & 1 & -27\\
                            0 & -4 & 2 & 2 & -6
                        \end{array}\right]
                    \]
                }
            \end{itemize}
        }
        \item {
            On the second iteration, $k = 2$.
            \begin{itemize}
                \item {
                    Relative pivots (we don't care about the first row anymore, so just three rows left):
                    \[ \biggl( \abs{\frac{2}{18}}, \abs{\frac{12}{13}}, \frac{4}{12} \biggr) \]
                    The second ratio is the largest, and it corresponds to the third row.
                }
                \item {
                    So, we swap row 3 with row $k = 2$.
                }
                \item {
                    Swap entries in $s$.
                }
                \item {
                    Forward elimination. Then backward substitution
                    on updated matrix as before.
                }
            \end{itemize}
        }
    \end{enumerate}
\end{example}

Remarks:
\begin{itemize}
    \item {
        In efficient implementations, the step of row swapping can be omitted, just 
        a permutation vector $l$ needs to be stored to keep track of matrix
        rearrangements.
        This will result in "echelon form" that will look like e.g.
        \[
            \begin{array}{c}
                2\ \to\\
                4\ \to\\
                1\ \to\\
                3\ \to
            \end{array}
            \left[\begin{array}{cccc|c}
                0 & * & * & * & *\\
                0 & 0 & 0 & * & *\\
                * & * & * & * & *\\
                0 & 0 & * & * & *
            \end{array}\right]
        \]
    }
    \item {
        GE with scaled partial pivoting always works when matrix is invertible, i.e.
        there exists a $A^{-1}$, such that $A A^{-1} = I$.
    
        It will fail for a singular (i.e. not invertible) matrix,
        because eventually a division by 0 will occur.
    }
    \item {
        Doing Gaussian elimination has computational complexity
        of $\mathcal{O}(n^3)$, because we have three nested for-loops.
        Cubic behaviour $n^3$ is problematic for large $n$!
    }
    \item {
        Traditionally, only the multiplication/division operations
        were counted in the number of operations $C$.
        (Since addition is very cheap).
        On present-day hardware, however, the costs are nearly as ``cheap''
        as addition or subtraction.
    }
    \item { 
        We are missing costs due to exchange with memory.
        Therefore, estimates of time complexity and reality may
        diverge substantially. 
    }
    \item {
        Backward substitution has order $n^2$, which does not affect the general 
        estimate of $n^3$.
    }
    \item {
        Scaled partial pivoting leads to an increase in cost, but order stays $n^3$.
    }
\end{itemize}