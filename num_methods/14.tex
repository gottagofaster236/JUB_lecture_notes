\pagebreak
\subsection{Piecewise polynomial interpolation}
Another remedy of the problems of high order polynomial interpolation is
\textit{piecewise polynomial interpolation}.

Subdivide $[a, b]$ into smaller pieces on which lower-order interpolation can be performed.
\begin{center}   
    \begin{tikzpicture}
        \begin{axis}[
            axis x line=left,
            axis y line=left,
            axis line style={-{Stealth[scale=1.5]}},
            xtick={0,1,2,3},
            xticklabels={$a$, $ $, $ $, $b$},
            ytick=\empty,
            xmin=-1, xmax=4, ymin=-1.5, ymax=1.5,
            unit vector ratio=1 1 1,
            samples=100,
        ]
            \tikzset{graph/.style={very thick, smooth, domain=0:3}}
            \tikzset{graph_point/.style={fill,circle,inner sep=1.5pt}}
            \addplot[blue, graph] {sin(150 * x)};
            \node[graph_point] at (0, 0){};
            \node[graph_point] at (1,  0.49999999999999994){};
            \node[graph_point] at (2, -0.8660254037844386){};
            \node[graph_point] at (3, 1.0){};
        \end{axis}
    \end{tikzpicture}
\end{center}
Problem: pieces of interpolations don't have the same derivatives on end points.
The solution is to use Hermite interpolation, in which
there are additional constraints on derivatives.

\subsubsection*{Spline interpolation}
\begin{definition}
    A function $s(u)$ a called a \textit{spline} of degree $k$ on the domain
    $[a, b]$ if:
    \begin{itemize}
        \item {
            $s \in C^{k - 1}([a, b])$ (it's $k - 1$ times continuously differentiable)
        }
        \item {
            There exist nodes $a = u_0 < u_1 < \dots < u_m = b$, such that
            $s|_{[u_i, u_{i+1}]}$ is a polynomial of degree $k$
            for $i = 0, \dots, m - 1$.
        }
    \end{itemize}
\end{definition}

\begin{example}
    A spline of degree 1 is such that $s \in C([a, b])$
    and $s|_{[u_i, u_{i+1}]}$ is linear.
    \begin{center}   
        \begin{tikzpicture}
            \begin{axis}[
                axis x line=left,
                axis y line=left,
                axis line style={-{Stealth[scale=1.5]}},
                xtick={0,1,2,3},
                xticklabels={$u_0$, $u_1$, $\dots$, $u_m$},
                ytick=\empty,
                xmin=-1, xmax=4, ymin=-1.5, ymax=1.5,
                unit vector ratio=1 1 1,
                samples=100,
            ]
                \tikzset{graph/.style={blue, very thick, smooth}}
                \tikzset{graph_point/.style={fill,circle,inner sep=1.5pt}}
                \addplot[graph, domain=0:1] {0.0 + 0.5 * x};
                \addplot[graph, domain=1:2] {1.8660254 - 1.3660254 * x};
                \addplot[graph, domain=2:3] {-4.59807621 + 1.8660254 * x};
                \node[graph_point] at (0, 0){};
                \node[graph_point] at (1,  0.49999999999999994){};
                \node[graph_point] at (2, -0.8660254037844386){};
                \node[graph_point] at (3, 1.0){};
            \end{axis}
        \end{tikzpicture}
    \end{center}

    A spline of degree 0 is just a piecewise constant function.
    \begin{center}   
        \begin{tikzpicture}
            \begin{axis}[
                axis x line=left,
                axis y line=left,
                axis line style={-{Stealth[scale=1.5]}},
                xtick={0,1,2,3},
                xticklabels={$u_0$, $u_1$, $\dots$, $u_m$},
                ytick=\empty,
                xmin=-1, xmax=4, ymin=-1.5, ymax=1.5,
                unit vector ratio=1 1 1,
                samples=100,
            ]
                \tikzset{graph/.style={blue, very thick, smooth}}
                \tikzset{graph_point/.style={fill,circle,inner sep=1.5pt}}
                \addplot[graph, domain=0:1] {0.0};
                \addplot[graph, domain=1:2] {0.49999999999999994};
                \addplot[graph, domain=2:3] {-0.8660254037844386};
                \node[graph_point] at (0, 0){};
                \node[graph_point] at (1,  0.49999999999999994){};
                \node[graph_point] at (2, -0.8660254037844386){};
                \node[graph_point] at (3, 1.0){};
            \end{axis}
        \end{tikzpicture}
    \end{center}
\end{example}
\begin{definition}
    A spline is in \textit{B-spline representation} (\textit{B}asis)
    if it is of the form
    \[ s(u) = \sum_{i=0}^m c_i N^n_i(u) \]
    where the $N_i^n(u)$ are the \textit{basis spline functions}
    of degree $n$ with minimal support
    (\textit{support} is the set where a function is non-zero).

    We define the basis splines (B-splines) $N_i^n(u)$ by recursion.
    Let $u_i$ be a set of nodes $u_0 < u_1 < \dots < u_m$.
    Then
    \[
        N_i^0(u) = \begin{cases}
            1 & \text{for } u_i \le u < u_{i + 1}\\
            0 & \text{else}
        \end{cases}
    \]
    and further
    \[
        N_{i}^n(u) = \alpha_i^{n-1}(u) N_i^{n - 1}(u) +
        \bigl(1 - \alpha_{i+1}^{n-1}(u)\bigr) N_{i + 1}^{n-1}(u)
        \text{ where }
        \alpha_i^{n-1}(u) =
        \frac{u - u_i}{u_{i+n} - u_{i}}
        \text{ is a local parameter}
    \]
\end{definition}
\begin{example}
    \mbox{}
    \begin{enumerate}
        \item {
            $n = 0$ (piecewise constant)
            \begin{center}   
                \begin{tikzpicture}
                    \begin{axis}[
                        axis x line=left,
                        axis y line=left,
                        axis line style={-{Stealth[scale=1.5]}},
                        xtick={0,1,2},
                        xticklabels={$u_i$, $u_{i+1}$, $u_{i+2}$},
                        ytick={1},
                        xmin=-1, xmax=4, ymin=0, ymax=1.5,
                        unit vector ratio=1 1 1,
                        samples=100,
                    ]
                        \tikzset{graph/.style={very thick, smooth}}
                        \addplot[red, graph, domain=0:1] {1};
                        \addlegendentry{$N_i^0(u)$}
                        \addplot[blue, graph, domain=1:2] {1};
                        \addlegendentry{$N_{i+1}^0(u)$}
                    \end{axis}
                \end{tikzpicture}
            \end{center}
        }
        \item {
            $n = 1$ (piecewise linear)
            \begin{center}   
                \begin{tikzpicture}
                    \begin{axis}[
                        axis x line=left,
                        axis y line=left,
                        axis line style={-{Stealth[scale=1.5]}},
                        xtick={0,1,2,3},
                        xticklabels={$u_i$, $u_{i+1}$, $u_{i+2}$, $u_{i+3}$},
                        ytick={1},
                        xmin=-1, xmax=5, ymin=0, ymax=1.5,
                        unit vector ratio=1 1 1,
                        samples=100,
                    ]
                        \tikzset{graph/.style={very thick, smooth}}
                        \addplot[red, graph, domain=0:2] {1 - abs(x - 1)};
                        \addlegendentry{$N_{i}^1(u)$}
                        \addplot[blue, graph, domain=1:3] {1 - abs(x - 2)};
                        \addlegendentry{$N_{i + 1}^1(u)$}
                    \end{axis}
                \end{tikzpicture}
            \end{center}
        }
        \item {
            $n = 2$ (piecewise quadratic)
            \begin{center}   
                \begin{tikzpicture}
                    \begin{axis}[
                        axis x line=left,
                        axis y line=left,
                        axis line style={-{Stealth[scale=1.5]}},
                        xtick={0,1,2,3,4},
                        xticklabels={$u_i$, $u_{i+1}$, $u_{i+2}$, $u_{i+3}$, $u_{i+4}$},
                        ytick={1},
                        xmin=-1, xmax=6, ymin=0, ymax=1.5,
                        unit vector ratio=1 1 1,
                        samples=100,
                    ]
                        \tikzset{graph/.style={very thick, smooth}}
                        \addplot[red, graph, domain=0:3] {
                            max(0, 1 - abs(x - 1)) * x / 2 +
                            max(0, 1 - abs(x - 2)) * (1 - (x - 1) / 2)
                        };
                        \addlegendentry{$N_{i}^2(u)$}
                        \addplot[blue, graph, domain=1:4] {
                            max(0, 1 - abs(x - 2)) * (x - 1) / 2 +
                            max(0, 1 - abs(x - 3)) * (1 - (x - 2) / 2)
                        };
                        \addlegendentry{$N_{i+1}^2(u)$}
                    \end{axis}
                \end{tikzpicture}
            \end{center}
        }
    \end{enumerate}
\end{example}

We have:
\begin{enumerate}
    \item {
        $\operatorname{supp} N_i^n = [u_i, u_{i+n+1}]$
    }
    \item {
        $N_i^n$ is a piecewise polynomial of degree $n$.
    }
    \item {
        $N_i^n$ are positive on $(u_i, u_{i+n+1})$, zero outside.
    }
    \item {
        $N_i^n$ have the partition of unity property:
        \[ \sum_{i=-n}^n N_i^n(u) = 1 \quad \forall u \in [a, b] \]
        \begin{center}   
            \begin{tikzpicture}
                \begin{axis}[
                    axis x line=left,
                    axis y line=left,
                    axis line style={-{Stealth[scale=1.5]}},
                    xtick={1,2,3},
                    xticklabels={$a=u_0$, $u_1$, $b=u_2$},
                    ytick={1},
                    xmin=0, xmax=5, ymin=0, ymax=1.5,
                    unit vector ratio=1 1 1,
                    samples=100,
                ]
                    \tikzset{graph/.style={very thick, smooth}}
                    \addplot[red, graph, domain=0:2] {1 - abs(x - 1)};
                    \addlegendentry{$N_{-1}^1$}
                    \addplot[green, graph, domain=1:3] {1 - abs(x - 2)};
                    \addlegendentry{$N_0^1$}
                    \addplot[blue, graph, domain=2:4] {1 - abs(x - 3)};
                    \addlegendentry{$N_1^1$}
                \end{axis}
            \end{tikzpicture}
        \end{center}
    }
\end{enumerate}

\subsubsection*{Interpolation with splines.}
Given nodes $u_0 < \dots < u_m$, values $p_0 < \dots < p_m$
find
\[ s(u) = \sum_{i=0}^m c_i N_i^n(u) \text{ such that } s(u_i) = p(i) \text{ for } i = 0,\dots, n \]
$n$ (degree) and $m$ (number of nodes minus one) can be independent.

Collocation matrix:
\[
    \Phi = \begin{bmatrix}
        N_0^n(u_0) & \dots & N_m^n(u_0)\\
        \vdots & \ddots & \vdots\\
        N_0^n(u_m) & \dots & N_m^n(u_m)
    \end{bmatrix}
    \in \mathbb{R}^{(m + 1) \times (m + 1)}
\]
We need to solve a linear problem to find $c_i$:
\[
    \Phi \vec{c} = \vec{p} \quad \vec{c} = (c_0, \dots, c_m) \in \mathbb{R}^{m+1},\
    \vec{p} = \mathbb{R}^{m+1}
\]
\begin{example}
    \textit{Cubic splines} interpolation, $n = 3$.
    Assume equidistant nodes $n_i = i + 2$.

    Find
    \[
        \sum_{i=0}^n c_i N_i^3(u) \coloneqq s(u)
    \]
    such that $s(u_i) = p_i$.
    \[
        \Phi = \begin{bmatrix}
            N_0^n(u_0) & \dots & N_m^3(u_0)\\
            \vdots & \ddots & \vdots\\
            N_0^3(u_m) & \dots & N_m^3(u_m)
        \end{bmatrix}
    \]
    Therefore:
    \begin{itemize}
        \item {
            $N_i^3(u_j) \ne 0$ when $j = i$ or $j = i + 1$ or $j = i - 1$.
        }
        \item {
            $N_i^3(u_j) = 0$, otherwise
        }
        \item {
            Also, $N_i^3(u_{i-1}) = N_i^3(u_{i+1})$.
        }
    \end{itemize}
    So, the collocation matrix is symmetric and tridiagonal!

    After computation, we find that
    $N_i^3(u) = \frac{2}{3}$ and $N_i^3(u_{i \pm 1}) = \frac{1}{6}$.

    Thus
    \[
        \Phi = \frac{1}{6}
        \begin{bmatrix}
            4 & 1 & & 0\\
            1 & \ddots & \ddots &\\
            & \ddots & \ddots & 1\\
            0 & & 1 & 4
        \end{bmatrix}
    \]
    The matrix is strictly positive definite, consequently a unique solution 
    to $\Phi \vec{c} = \vec{p}$ exists.
\end{example}
