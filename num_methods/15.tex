\subsection{Least squares approximation}

Instead of doing an actual interpolation, we can look for an approximation, i.e. a curve that stays
``close'' to the measurements and is of certain (function) type.

Polynomial interpolation is restrained to polynomials. In least squares
approximation, we are free to choose the types of (function) type.

\begin{example}
    Surface tension $S$ of a liquid depends on the liquid's temperature. It's known that the dependence is linear, i.e.
    \[ S = S(T) = a + bt \]
    for some $a, b \in \mathbb{R}$. In order to find $a$ and $b$ for a specific liquid we do measurements.
    \begin{center}   
        \begin{tikzpicture}
            \begin{axis}[
                axis x line=left,
                axis y line=left,
                axis line style={-{Stealth[scale=1.5]}},
                xtick={0,1,2,3},
                xticklabels={$T_0$, $T_1$, $T_2$, $T_3$},
                ytick=\empty,
                xmin=-0.5, xmax=3.5, ymin=-0.5, ymax=3.5,
                unit vector ratio=1 1 1,
                samples=100,
                xlabel={$T$},
                ylabel={$S$},
            ]
                \tikzset{graph/.style={very thick, smooth, domain=0:3}}
                \tikzset{graph_point/.style={fill,circle,inner sep=1.5pt}}
                \addplot[blue, graph] {x};
                \node[graph_point] at (0, -0.3){};
                \node[graph_point] at (1,  0.7){};
                \node[graph_point] at (2, 2.3){};
                \node[graph_point] at (3, 2.7){};
            \end{axis}
        \end{tikzpicture}
    \end{center}
    No straight line currently interpolates all the points.
    Of all lines that could be considered a fit, which is the best?
    It's an optimization problem, we want to minimize the error.

    Consider approximation error at nodes and minimize.
    \begin{itemize}
        \item {
            Suppose we have $n + 1$ measurements $s_0, \dots, s_n$
            for nodes $T_0, \dots, T_n$.
        }
        \item {
            At any node $T_i$ the error is
            \[ \abs{s_i - (a T_i + b)} \]
        }
        \item {
            The sum of all local errors is (for fixed $a, b$):
            \[ E_1(a, b) = \sum_{i=0}^n \abs{s_i - (a T_i + b)} \]
            Here the \text{approximation error} is measured on $l_1$-norm.
        }
        \item {
            To find the minimum we need to find critical points (i.e. roots of the derivative)
            of $E_1(a, b)$. But $E_1(a, b)$ is not continuously differentiable
            for all $a, b$, because we have a modulus function that is not differentiable at zero.
        }
        \item {
            \textbf{Remedy:} go over to squares of local errors.
        }
        \item {
            The approximation error in the $l_2$-norm:
            \[ E_1(a, b) = \sum_{i=0}^n \bigl(s_i - (a T_i + b)\bigr)^2 \]
            So now, to find the critical points we consider
            \begin{align*}
                \frac{\partial E_2}{\partial a} &=
                \frac{\partial}{\partial a} \Bigl(
                    \sum_{i=0}^n \bigl(s_i - (aT_i + b)\bigr)^2
                \Bigr)\\
                &= \sum \frac{\partial}{\partial a} \bigl(
                    s_i - (aT_i + b)
                \bigr)^2 = -\sum 2 \bigl(s_i - (aT_i + b)\bigr)T_i\\
                &= -\sum 2 s_i T_i + 2a \sum T_i^2 + 2b \sum T_i
            \end{align*}
            \text{Critical point:}
            \[ 
                0 = \frac{\partial E_2}{\partial a} \iff 
                a \sum T_i^2 + b \sum T_i = \sum s_i T_i \quad (1)
            \]
            And
            \begin{align*}
                \frac{\partial E_2}{\partial b} &=
                \frac{\partial}{\partial b} \Bigl(\sum \bigl(
                    s_i - (a T_i + b)
                \bigr)^2\Bigr)\\
                &= -2 \sum \bigl(s_i - (aT_i + b)\bigr)
            \end{align*}
            The critical point:
            \begin{align*}
                0 = \frac{\partial E_2}{\partial b} 
                &\iff a \sum T_i + b \sum 1 = \sum S_i \\
                &\iff a \sum T_i + b(n + 1) = \sum S_i \quad (2)
            \end{align*}
            (1) and (2) determine a linear system of equations in $a$ and $b$:
            \[
                \begin{bmatrix}
                    \sum T_i^2 & \sum T_i\\
                    \sum T_i & (n + 1)
                \end{bmatrix}
                \begin{pmatrix}
                    a \\ b
                \end{pmatrix} = \begin{bmatrix}
                    \sum S_i T_i\\ \sum S_i
                \end{bmatrix}
            \]
            To solve this we just have to invert a $2 \times 2$ matrix.
            Note that this is the matrix for fitting a linear problem, it's 
            gonna be different for different problems!
        }
    \end{itemize}
\end{example}

\textbf{In more generality:}
The least squares approximation assumes that we have measurements $y_i,\ i = 0,\dots, n$
at nodes $x_i,\ i = 0,\dots, n$. We are lookiing for a function
\[
    y(x) = \sum_{j=0}^m c_j g_j(x)
\]
for linearly independent functions $g_j(x)$ (otherwise one function could be replaced
with a linear combination of others).

In the least squares approach we are minimizing the $l_2$-error:
\[
    \varphi(c_0, \dots, c_m) = \sum_{i = 0}^n \Bigl(
        y_i - \sum_{j=0}^m c_j g_j(x_i)
    \Bigr)^2
\]
A necessary condition is vanishing partial derivatives:
\[ \frac{\partial \varphi}{\partial c_k} = 0 \text{ for } k = 0, \dots, m\]
In fact:
\begin{align*}
    &\frac{\partial \varphi}{\partial c_k} =
    -\sum_{i=0}^n 2 \Bigl(y_i - \sum_{j=0}^m c_j g_j(x_i)\Bigr) g_k(x_i) \overset{!}{=} 0
    \\&\iff \sum_{j=0}^m \Bigl(
        \sum_{i=0}^n g_j(x_i) g_k(x_i)
    \Bigr) c_j = \sum_{i=0}^n y_i g_k(x_i)
    \text{ for } k = 0,\dots, m
\end{align*}
These equations are called \textit{normal equations}.
They yield an $(m + 1) \times (m + 1)$ system.
We need $n \ge m$ in order to have a unique solution (i.e.,
the number of data points shouldn't be less than the number of free variables).
Otherwise, the system will be linearly dependent.

\textbf{General examples:}
\begin{itemize}
    \item {
        Drag force is proportional to wind speed squared.
    }
    \item {
        Deformation of a spring is linear function of applied force.
    }
\end{itemize}

\begin{example}
    $y(x) = a \log(x) + b\cos(x)$.
    Write the \textit{normal equations}.

    Assume nodes $x_i$ and values $y_i$ given for
    $i = 0, \dots, n$. Error in $l_2$-norm:
    \[
        E_2(a, b) = \sum_{i=0}^n \bigl(y_i - y(x)\bigr)^2 = 
        \sum_{i=0}^n (y_i - a\log(x) - b\cos(x_i))^2
    \]
    necessary conditions:
    \[
        \frac{\partial E_2}{\partial a} = \frac{\partial E_2}{\partial b} = 0
    \]
    The second derivative is not needed, because the function is convex in $a$ and $b$
    (it's a quadratic function with a positive coefficient before $x^2$).
    Therefore, the only critical point is the minimum.
    \begin{align*}
        &\frac{\partial E_2}{\partial a} =
        \sum 2\bigl(a \log(x_i) + b\cos(x_i) - y_i\bigr) \log(x_i)\\
        &\frac{\partial E_2}{\partial a} \overset{!}{=} \iff
        a \sum \log(x_i)^2 + b\sum \cos(x_i) \log(x_i) = 
        \sum y_i \log(x_i)
    \end{align*}
    In the same way, we can derive that
    \[
        \frac{\partial E_2}{\partial b} = 0 \iff
        a \sum \log(x_i) \cos(x_i) + b\sum \cos^2(x_i) = 
        \sum y_i \cos(x_i)
    \]
    This linear system of equations can be formulated as 
    \begin{align*}
        &\Phi \Phi^\intercal \begin{pmatrix}
            a \\ b
        \end{pmatrix} = \Phi \vec{y}
        \\&
        \Phi = \begin{bmatrix}
            \log(x_0) & \dots & \log(x_n)\\
            \cos(x_0) & \dots & \cos(x_n)
        \end{bmatrix}
    \end{align*}
    $\Phi^\intercal$ is the \textit{collocation matrix}.
    For the exam, it's much quicker
    to memorize the collocation matrix and use this formula,
    instead of remembering how to get to those equations by hand.
    \[
        \begin{bmatrix}
            \log(x_0) & \dots & \log(x_n)\\
            \cos(x_0) & \dots & \cos(x_n)
        \end{bmatrix}
        \begin{bmatrix}
            \log(x_0) & \cos(x_0) \\
            \vdots & \vdots\\
            \log(x_n) & \cos(x_n)
        \end{bmatrix}
        \begin{pmatrix}
            a \\ b
        \end{pmatrix} = 
        \begin{bmatrix}
            \log(x_0) & \dots & \log(x_n)\\
            \cos(x_0) & \dots & \cos(x_n)
        \end{bmatrix} 
        \begin{pmatrix}
            y_0 \\ \vdots \\ y_n
        \end{pmatrix}
    \]
\end{example}